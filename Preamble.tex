%TODO 等的使用
%%TODO 内容问题
%%HACK 粗鄙技巧
%%CODE 代码改进

%TODO——注意事项

\documentclass[oneside]{book}


%TODO——宏包
\ifdefined \TEST \else
%\fi 在 字体 之前

%%页面尺寸
\usepackage{geometry}
	\geometry{
		a4paper,
		left = 2.54 cm, right = 2.54 cm, top = 3.18 cm, bottom = 3.18 cm,
		headheight = 3 cm
	}

%%设置标题
\usepackage{titlesec}

%%交叉引用,超链接等
\usepackage[hyperindex]{hyperref}
	\hypersetup{
		%  PDF 书签
		bookmarksopen = true,
		bookmarksopenlevel = 1,
		bookmarksnumbered = true,
		%  PDF 标题作者
%		pdftitle = {LaTeX 排版笔记},
%		pdfauthor = {曾祥东},
		%脚注
%		hyperfootnotes = false,
		%目录  只引用页码
		linktoc = page,
		%超链接颜色
		colorlinks,
		linkcolor = {red!60!black},
		citecolor = {green!50!black},
		urlcolor = {blue!70!black}
	}
\fi

%%常规字体选择
%
%正文字体:Palatino
%无衬线字体A:Helvetica
%无衬线字体B:Source Sans Pro
%等宽字体:Courier
%
%以下两种方式已弃用:
%Helvetica: \usefont{T1}{phv}{m}{n}
%Courier: \usefont{T1}{pcr}{m}{n}
%
\usepackage[no-math]{fontspec}
	\setmainfont[
		Extension = .otf,
		BoldFont = texgyrepagella-bold,
		ItalicFont = texgyrepagella-italic,
		SlantedFont = texgyrepagella-italic
	]{texgyrepagella-regular}
	\setsansfont[
		Extension = .otf,
		BoldFont = texgyreheros-bold,
		ItalicFont = texgyreheros-italic,
		SlantedFont = texgyreheros-italic
	]{texgyreheros-regular}
	\setmonofont[
		Extension = .otf,
		BoldFont = texgyrecursor-bold,
		ItalicFont = texgyrecursor-italic,
		SlantedFont = texgyrecursor-italic,
		Ligatures = NoCommon
	]{texgyrecursor-regular}
	\newfontfamily{\SourceSans}{Source Sans Pro}
	\newfontfamily{\Songti}{方正书宋_GBK}

%%AMS 数学支持
\usepackage{amsmath}

%%Pi 符号
\usepackage{pifont}
	% 正确 √
	\newcommand{\cmark}{\ding{51}}
	% 错误 ×
	\newcommand{\xmark}{\ding{55}}

%%调用 Unicode OpenType 数学字体
\usepackage{unicode-math}
	\setmathfont[
		math-style = ISO,
		bold-style = ISO
	]{texgyrepagella-math.otf}
%加粗使用 \symbf{}
%直立希腊字母:\uppi 等

%%中文文字处理
\usepackage[UTF8, heading = true]{ctex}
	\pagestyle{empty}
\usepackage{xeCJK}
	\setCJKmainfont[
		BoldFont = 方正黑体_GBK,
		ItalicFont = 方正楷体_GBK,
		Mapping = fullwidth-stop
	]{方正书宋_GBK}
	\setCJKsansfont[
		BoldFont = 方正黑体_GBK,
		ItalicFont = 方正黑体_GBK,
		Mapping = fullwidth-stop
	]{方正黑体_GBK}
%	\setCJKsansfont[
%		BoldFont = 思源黑体 CN Regular,
%		ItalicFont = 思源黑体 CN Regular,
%		Mapping = fullwidth-stop
%	]{思源黑体 CN Regular}
	\setCJKmonofont[
		BoldFont = 方正仿宋_GBK,
		ItalicFont = 方正楷体_GBK,
		Mapping = fullwidth-stop
	]{方正仿宋_GBK}
	\setCJKfamilyfont{宋体}{方正书宋_GBK}
	\setCJKfamilyfont{楷体}{方正楷体_GBK}
	\setCJKfamilyfont{黑体}{方正黑体_GBK}
	\setCJKfamilyfont{仿宋}{方正仿宋_GBK}


%%脚注增强版
\usepackage[stable, perpage, bottom]{footmisc}
	%需要调用 pifont 宏包
	%衬线加圈阳文数字:\ding{172}~\ding{181} (1~10)
	%无衬线加圈阳文数字:\ding{192}~\ding{201} (1~10)
	%xTODO:20160705 脚注不用上标
	%HACK:20160709 见 http://tex.stackexchange.com/questions/19844/how-to-set-superscript-footnote-mark-in-the-text-body-but-normalsized-in-the-foo
	\renewcommand{\thefootnote}{\ding{\numexpr191+\value{footnote} } }
	\makeatletter
	\newlength{\fnBreite}
	\renewcommand{\@makefntext}[1]{%
		\settowidth{\fnBreite}{\footnotesize\@thefnmark.i}
		\protect\footnotesize\upshape%
		\setlength{\@tempdima}{\columnwidth}\addtolength{\@tempdima}{-\fnBreite}%
		\makebox[\fnBreite][l]{\@thefnmark\phantom{  }}%
%		\parbox[t]{\@tempdima}{\everypar{\hspace*{1em}}\hspace*{-1em}\upshape#1}
		}
	\makeatother

%%颜色
\usepackage[svgnames]{xcolor}

%%图形
\usepackage{graphicx}

%%绘图
\usepackage{tikz}
	\usepgflibrary{arrows.meta}
	\tikzset{>=Stealth}

%%TeX Logos
\ifdefined \TEST \else
\usepackage{hologo}
\fi

%%物理、数学符号
\usepackage{physics}

%%单位
\usepackage{siunitx}

\ifdefined \TEST \else
%\fi 在 TODO——命令 之前

%%定制列表环境
\usepackage{enumitem}
	%定义带缩进的左对齐格式
	%前一个参数 2.1 em 确定标签的位置
	%后一个参数 1.2 em 确定标签与文字的距离
	%HACK:20160709 可能与字体、字号、标签内容有关
	\SetLabelAlign{leftalignwithindent}{\hspace{2.1 em} \makebox[1.2 em][l]{#1}}

%%定理类环境
%\usepackage[thmmarks, amsmath]{ntheorem}
%	\theoremstyle{plain} %带编号
%%	\theoremheaderfont{\myHeavy}
%	\theorembodyfont{\normalfont}
%%	\theoremsymbol{}
%%	\theoremsymbol{\ensuremath{\triangleleft}}	
%		\newtheorem{_myQuestion}{例}[chapter]
%		\newtheorem{_myAnswer}{答}[chapter]

%%索引
\usepackage{imakeidx}
	\makeindex[
		name = pkg,
		title = {宏包索引}
	]
	\makeindex[
		name = cmd,
		title = {命令、选项索引}
	]
	\newcommand{\BH}[1]{\hyperpage{#1}}
%	\let \oldindex = \index
%	\renewcommand{\index}[1]{\oldindex{#1|BH}}
	\newcommand{\indexpkg}[1]{\index[pkg]{#1@\pkg{#1}|BH} }
	\newcommand{\indexcmd}[1]{\index[cmd]{#1@\texttt{#1}|BH} }


%\usepackage[numbers]{natbib}%\raggedleft
%\bibliographystyle{mybst}
%网址右对齐的实现
%FIXME:20160712 \CTeX logo 在本文档字体下倾斜时重叠
%\providecommand{\url}[1]{\texttt{#1}}
%\providecommand{\urlprefix}{\hspace*{\fill}}

%TODO——环境
%%定制列表(不编号)
\newenvironment{myItemize}
	{\begin{enumerate} [
		label=\bullet, %标签样式:圆点
		align = leftalignwithindent, %对齐(见上)
		listparindent = 2 em, %条目段落缩进
		leftmargin = 0 pt, %文字左边距
		topsep = 0 pt,
		itemsep = 0 pt,
		parsep = 0 pt
	]}
	{\end{enumerate}}
\fi

%TODO——命令
%%空行
\newcommand{\blankline}{\mbox{}}
%强调
\newcommand{\emphA}[1]{{\bfseries #1}}
\newcommand{\emphB}[1]{{\itshape #1}}
%%数集黑板粗体
\newcommand{\setbb}[1]{\symbb{#1}}
%%哥特体
\newcommand{\gothic}[1]{\symfrak{#1}}
\newcommand{\domainB}{\gothic{B}} %邻域
\newcommand{\domainD}{\gothic{D}} %定义域
%%向量粗体
\renewcommand{\vb}[1]{\symbf{#1}}


%TODO——标题页
\title{
	\vspace{-4 cm} \color{Sienna} \Huge 微积分的深化
}
\author{
	\CJKfamily{楷体} \color{DarkRed} \Large 复旦大学\phantom{空格}谢锡麟
}
\date{
	\CJKfamily{楷体} \color{Goldenrod} \Large \today
}