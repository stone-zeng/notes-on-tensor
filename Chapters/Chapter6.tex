\section{Frenet 标架}
%CODE 20161130 TeX Studio bug 无法换行
% #1898 Freeze while typing \texorpdfstring
% https://sourceforge.net/p/texstudio/bugs/1898/
\subsection{\texorpdfstring{$\Rm$}{R\^{}m} 空间中曲线的表示}
$\Rm$ 空间中的\emphA{曲线},就是一个\emphB{单参数}的向量值映照:
\begin{equation}
	\mmap{\V{X}(t)}{[\alpha,\,\beta]\ni t}
		{\V{X}(t)=\mqty[X^1(t) \\ \vdots \\ X^m(t)]\in\Rm} \fullstop
\end{equation}
考虑该向量值映照关于参数 $t$ 的变化率
\begin{equation}
	\dot{\V{X}}(t)\coloneq\dv{\V{X}}{t} (t)
	=\lim_{\incr t\to 0} \frac{\V{X}(t+\incr t)-\V{X}(t)}{\incr t}
	\eqcolon \mqty[\dot{X}^1(t) \\ \vdots \\ \dot{X}^m(t)] \comma
\end{equation}
按照物理上的习惯,我们用点表示对 $t$ 的导数。
若该极限存在,则称 $\V{X}(t)\in\Rm$ 在点 $t$
处\emphA{可微}。此时,$\dot{\V{X}}(t)$ 称为曲线
$\V{X}(t)$ 的\emphA{切向量}。上述极限可以等价地表述为
\begin{equation}
	\V{X}(t+\incr t) = \V{X}(t)+\dv{\V{X}}{t} (t) \cdot\incr t
		+\sOv{\incr t} \fullstop
\end{equation}
在 $t_0$ 处,则可以写成
\begin{equation}
	\V{X}(t) =\V{X}\qty(t_0)+\dv{\V{X}}{t} \qty(t_0)
		\cdot\qty(t-t_0)+\sOv{t-t_0} \fullstop
\end{equation}
该方程表示一条直线,称为曲线 $\V{X}(t)$
在 $t_0$ 处的\emphA{切线}。

在物理域中,\emphA{弧长} $s$ 可以表示为
\begin{equation}
	s=\int_\alpha^\beta \norm{\dv{\V{X}}{t} (t)} \dd{t} \comma
\end{equation}
两边对 $t$ 求导,可有
\begin{equation}
	\dv{s}{t} (t) = \norm{\dv{\V{X}}{t} (t)} \fullstop
\end{equation}
为了研究问题的方便,我们以后将用弧长作为曲线的参数,即
\begin{equation}
	\mmap{\V{r}(s)}{[0,\,L]\ni s}{\V{r}(s)\in\Rm} \fullstop
\end{equation}
对应的切向量为
\begin{equation}
	\dot{\V{r}}(s)\coloneq\dv{\V{r}}{s} (s)
	=\lim_{\incr s\to 0} \frac{\V{r}(s+\incr s)-\V{r}(s)}{\incr s}
	\fullstop
\end{equation}
根据链式法则,
\begin{equation}
	\dv{\V{r}}{s} (s)
	=\dv{\V{r}}{t} (t) \cdot \dv{t}{s} (s)
	=\frac*{\dv{\V{r}}{t} (t)}{\dv{s}{t} (t)}
	=\frac*{\dv{\V{r}}{t} (t)}{\norm{\dv{\V{r}}{t} (t)}} \comma
\end{equation}
因而 $\dot{\V{r}}(s)$ 是一个单位向量,即
\begin{equation}
	\norm{\dot{\V{r}}(s)}=1 \fullstop
\end{equation}
接下来继续对 $\dot{\V{r}}(s)$ 求导:
\begin{equation}
	\ddot{\V{r}}(s)\coloneq\dv{\dot{\V{r}}}{s} (s) \fullstop
\end{equation}
由于 $\norm{\dot{\V{r}}(s)}=1$,因此
\begin{equation}
	1=\norm{\dot{\V{r}}(s)}^2
	=\ipb{\vphantom{0^0} \dot{\V{r}}(s)}{\dot{\V{r}}(s)} \comma
\end{equation}
两边求导,则有
\begin{equation}
	0=\ipb{\vphantom{0^0} \ddot{\V{r}}(s)}{\dot{\V{r}}(s)}
		+\ipb{\vphantom{0^0} \dot{\V{r}}(s)}{\ddot{\V{r}}(s)}
	=2\cdot\ipb{\vphantom{0^0} \ddot{\V{r}}(s)}{\dot{\V{r}}(s)}
	\fullstop
\end{equation}
内积为零,就意味着正交:
\begin{equation}
	\ddot{\V{r}}(s) \perp \dot{\V{r}}(s) \fullstop
\end{equation}

现在我们把目光限定在 $\realR^3$ 空间中。将 $\dot{\V{r}}(s)$
和 $\ddot{\V{r}}(s)$ 分别记为 $\V{\tau}(s)$ 和 $\V{k}(s)$。
如前所述,$\V{\tau}(s)$ 已经是单位向量;
而 $\V{k}(s)$ 仍需作单位化处理,其结果记作 $\V{n}(s)$,即
\begin{equation}
	\V{n}(s)\coloneq\frac{\V{k}(s)}{\norm[\realR^3]{\V{k}(s)}}
	=\frac{\ddot{\V{r}}(s)}{\norm[\realR^3]{\ddot{\V{r}}(s)}}
	\fullstop
\end{equation}
最后,只要再令
\begin{equation}
	\V{b}(s)\coloneq\V{\tau}(s)\cp\V{n}(s) \comma
\end{equation}
我们便有了 $\realR^3$ 空间中的一组\emphB{单位正交基}:
\begin{equation}
	\qty\big{\V{\tau}(s),\,\V{n}(s),\,\V{b}(s)}
	\subset\realR^3 \comma
\end{equation}
它们称作\emphA{Frenet 标架}。

\subsection{标架运动方程}
考虑 Frenet 标架关于弧长参数 $s$ 的变化率,即\emphB{标架运动方程}:
\begin{equation}
	\qty\big{\dot{\V{\tau}}(s),\,\dot{\V{n}}(s),\,\dot{\V{b}}(s)}
	\subset\realR^3 \fullstop
\end{equation}

为此,我们需要先给出一个引理:设 $\qty{\V{e}_i(t)}^m_{i=1}$
是 $\Rm$ 空间中的一组\emphB{活动单位正交基},它们满足
\begin{equation}
	\ipb{\V{e}_i(t)}{\V{e}_j(t)}=\KroneckerDelta*{ij} \fullstop
	\label{eq:活动单位正交基引理_单位正交性}
\end{equation}
这组基的导数仍位于 $\Rm$ 空间,用自身展开,可有
%\begin{equation}
%	\dot{\V{e}}_i(t) = P_{ij}\V{e}_j(t) \fullstop
%\end{equation}
%写成矩阵形式为
\begin{equation}
	\mqty[\dot{\V{e}}_1(t),\,\cdots,\,\dot{\V{e}}_m(t)]
	=\mqty[\V{e}_1(t),\,\cdots,\,\V{e}_m(t)] \Mat{P}(t) \fullstop
	\label{eq:活动单位正交基引理_导数}
\end{equation}
此时,我们有
\begin{equation}
	\Mat{P}(t)\in\Skw \comma
\end{equation}
即 $\Mat{P}(t)$ 是一个\emphB{反对称矩阵}。

\begin{myProof}
对式~\eqref{eq:活动单位正交基引理_单位正交性} 两边求导,得
\begin{equation}
	\ipb{\dot{\V{e}}_i(t)}{\V{e}_j(t)}
	+\ipb{\V{e}_i(t)}{\dot{\V{e}}_j(t)} = 0 \in\realR \comma
\end{equation}
写成矩阵形式,为
\begin{equation}
	\mqty[\dot{\V{e}}_1\trans(t) \\ \vdots \\ \dot{\V{e}}_m\trans(t)]
	\mqty[\V{e}_1(t),\,\cdots,\,\V{e}_m(t)]
	+\mqty[\V{e}_1\trans(t) \\ \vdots \\ \V{e}_m\trans(t)]
	\mqty[\dot{\V{e}}_1(t),\,\cdots,\,\dot{\V{e}}_m(t)]
	=\Mat{0}\in\realR^{m\times m} \fullstop
\end{equation}
引入矩阵 $\Mat{E}=\qty[\V{e}_1(t),\,\cdots,\,\V{e}_m(t)]$,
则上式与 \eqref{eq:活动单位正交基引理_导数}~式可以分别表示成
\begin{equation}
	\dot{\Mat{E}}\trans\Mat{E}+\Mat{E}\trans\dot{\Mat{E}}
	=\Mat{0}\in\realR^{m\times m}
\end{equation}
和
\begin{equation}
	\dot{\Mat{E}} = \Mat{E}\Mat{P}\in\realR^{m\times m} \fullstop
\end{equation}
两式联立,可有
\begin{align}
	\Mat{0}=\dot{\Mat{E}}\trans\Mat{E}
		+\Mat{E}\trans\dot{\Mat{E}} \notag \\
	&=\qty\big(\Mat{E}\Mat{P})\trans \Mat{E}
		+\Mat{E}\trans \qty\big(\Mat{E}\Mat{P}) \notag \\
	&=\Mat{P}\trans \qty\big(\Mat{E}\trans\Mat{E})
		+\qty\big(\Mat{E}\trans\Mat{E}) \Mat{P}
	=\Mat{P}\trans+\Mat{P} \comma
\end{align}
即 $\Mat{P}\trans=-\Mat{P}$。\myPROBLEM{按照定义},便知
$\Mat{P}(t)\in\Skw$。
\end{myProof}