\section{张量的范数}
\subsection{赋范线性空间}
对于一个\emphA{线性空间} $\SPACE{V}$,它总是定义了\emphA{线性结构}:
\begin{equation}
	\forall\, \V{x},\,\V{y}\in\SPACE{V}
	\text{\ 和\ } \forall\, \alpha,\,\beta\in\realR,\quad
	\alpha\,\V{x}+\beta\,\V{y} \in \SPACE{V} \fullstop
\end{equation}
为了进一步研究的需要,我们还要引入\emphA{范数}的概念。
所谓“范数”,就是对线性空间中任意元素\emphB{大小}的一种刻画。
举个我们熟悉的例子, $m$ 维 Euclid 空间 $\Rm$ 中某个向量的范数,
就定义为该向量在 Descartes 坐标下各分量的平方和的平方根。

一般而言,线性空间 $\SPACE{V}$ 中的范数
$\norm[\SPACE{V}]{\cdotord}$ 是从 $\SPACE{V}$ 到 $\realR$
的一个映照,并且需要满足以下三个条件:

\begin{myEnum}
\item \emphA{非负性}
\begin{equation}
	\forall\,\V{x}\in\SPACE{V},\quad
	\norm[\SPACE{V}]{\V{x}} \geqslant 0
\end{equation}
以及\emphA{非退化性}
\begin{equation}
	\forall\,\V{x}\in\SPACE{V},\quad
	\norm[\SPACE{V}]{\V{x}}=0
	\iff \V{x}=\V{0}\in\SPACE{V} \comma
\end{equation}
这里的 $\V{0}$ 是线性空间 $\SPACE{V}$ 中的\emphA{零元素},
它是唯一存在的。

\blankline

\item 由于零元是唯一的,因此线性空间中的元素 $\V{x}$
就与从 $\V{0}$ 指向它的向量一一对应。
因此,线性空间中的元素也常被称为“向量”。

考虑线性空间中的数乘运算。从几何上看, $\V{x}$ 乘上 $\lambda$,
就是将 $\V{x}$ 沿着原来的指向进行伸缩。显然有
\begin{equation}
	\forall\,\V{x}\in\SPACE{V}
	\text{\ 和\ } \forall\,\lambda\in\realR,\quad
	\norm[\SPACE{V}]{\lambda\,\V{x}}
	=\abs{\lambda}\cdot\norm[\SPACE{V}]{\V{x}} \comma
\end{equation}
这称为\emphA{正齐次性}。

\myPROBLEM{想要图吗?}

\item 线性空间中的加法满足\emphB{平行四边形法则}。直观地看,就有
\begin{equation}
	\forall\, \V{x},\,\V{y}\in\SPACE{V},\quad
	\norm[\SPACE{V}]{\V{x}+\V{y}} \leqslant
	\norm[\SPACE{V}]{\V{x}}+\norm[\SPACE{V}]{\V{y}} \comma
\end{equation}
这称为\emphA{三角不等式}。
\end{myEnum}

定义了范数的线性空间称为\emphA{赋范线性空间}。

\subsection{张量范数的定义}
考虑 $p$ 阶张量 $\T{\Phi}\in\Tensors{p}$,
它可以用\emphB{逆变分量}或\emphB{协变分量}来表示:
\begin{braceEq*}{\T{\Phi}=}
	\Phi^{i_1 \cdots i_p}\,
		\V{g}_{i_1}\tp\cdots\tp\V{g}_{i_p} \comma \\
	\Phi_{i_1 \cdots i_p}\,
		\V{g}^{i_1}\tp\cdots\tp\V{g}^{i_p} \comma
\end{braceEq*}
其中
\begin{braceEq}
	\Phi^{i_1 \cdots i_p}&=
		\T{\Phi}\qty(\V{g}^{i_1},\,\cdots,\,\V{g}^{i_p}) \fullstop \\
	\Phi_{i_1 \cdots i_p}&=
		\T{\Phi}\qty\big(\V{g}_{i_1},\,\cdots,\,\V{g}_{i_p}) \comma
\end{braceEq}
张量的\emphA{范数}定义为
\begin{equation}
	\norm[\Tensors{p}]{\T{\Phi}}\defeq
	\sqrt{\Phi^{i_1 \cdots i_p} \, \Phi_{i_1 \cdots i_p}}
	\in\realR \fullstop
\end{equation}
$i_1 \cdots i_p$ 可独立取值,每个又有 $m$ 种取法,
所以根号下共有 $m^p$ 项。
注意 $\Phi^{i_1 \cdots i_p}$ 与 $\Phi_{i_1 \cdots i_p}$
未必相等,因而根号下的部分未必是平方和,
这与 Euclid 空间中向量的模是不同的。

复习一下 \ref{subsec:相对不同基的张量分量之间的关系}~小节,
我们可以用另一组(带括号的)基表示张量 $\T{\Phi}$:
\begin{braceEq}
	\Phi^{i_1 \cdots i_p} &=
		c^{i_1}_{(\xi_1)} \cdots c^{i_p}_{(\xi_p)} \,
		\Phi^{(\xi_1)\cdots(\xi_p)} \comma \\
	\Phi_{i_1 \cdots i_p} &=
		c^{(\eta_1)}_{i_1} \cdots c^{(\eta_p)}_{i_p} \,
		\Phi_{(\eta_1)\cdots(\eta_p)} \comma
\end{braceEq}
其中的 $c^i_{(\xi)}=\ipb{\V{g}_{(\xi)}}{\V{g}^i}$,
$c^{(\eta)}_i=\ipb{\V{g}^{(\eta)}}{\V{g}_i}$,它们满足
\begin{equation}
	c^i_{(\xi)}\,c^{(\eta)}_i
	=\KroneckerDelta{(\eta)}{(\xi)} \fullstop
\end{equation}
于是
\begin{align}
	&\alspace\Phi^{i_1 \cdots i_p} \, \Phi_{i_1 \cdots i_p} \notag \\
	&=\qty(c^{i_1}_{(\xi_1)} \cdots c^{i_p}_{(\xi_p)} \,
			\Phi^{(\xi_1)\cdots(\xi_p)})
		\qty(c^{(\eta_1)}_{i_1} \cdots c^{(\eta_p)}_{i_p} \,
			\Phi_{(\eta_1)\cdots(\eta_p)}) \notag \\
	&=\qty(c^{i_1}_{(\xi_1)} c^{(\eta_1)}_{i_1}) \cdots
		\qty(c^{i_p}_{(\xi_p)} c^{(\eta_p)}_{i_p}) \,
		\Phi^{(\xi_1)\cdots(\xi_p)} \Phi_{(\eta_1)\cdots(\eta_p)}
		\notag \\
	&=\KroneckerDelta{(\eta_1)}{(\xi_1)} \cdots
		\KroneckerDelta{(\eta_p)}{(\xi_p)} \,
		\Phi^{(\xi_1)\cdots(\xi_p)} \Phi_{(\eta_1)\cdots(\eta_p)}
		\notag \\
	&=\Phi^{(\xi_1)\cdots(\xi_p)} \Phi_{(\xi_1)\cdots(\xi_p)}
	\fullstop
\end{align}
它是 $\T{\Phi}$ 在另一组基下的逆变分量与协变分量乘积之和。

以上结果说明,张量的范数不依赖于基的选取,
这就好比用不同的秤来称同一个人的体重,都将获得相同的结果。
既然如此,不妨采用\emphB{单位正交基}来表示张量的范数:
\begin{align}
	\norm[\Tensors{p}]{\T{\Phi}}
	&\defeq\sqrt{\Phi^{i_1 \cdots i_p} \, \Phi_{i_1 \cdots i_p}}
	\notag \\
	&=\sqrt{\Phi^{\orthIdx{i_1}\cdots\orthIdx{i_p}} \,
		\Phi_{\orthIdx{i_1}\cdots\orthIdx{i_p}}} \notag \\
	&\eqcolon\sqrt{\sum\nolimits_{i_1,\,\cdots,\,i_p=1}^{m}
		\qty\big(\Phi\midscript{\orthIdx{i_1,\,\cdots,\,i_p}})^2}
	\fullstop
\end{align}
这里的 $\Phi\midscript{\orthIdx{i_1,\,\cdots,\,i_p}}$
表示张量 $\T{\Phi}$ 在单位正交基下的分量,它的指标不区分上下。

有了这样的表示,很容易就可以验证张量范数符合之前的三个要求。
一组数的平方和开根号,必然是\emphB{非负}的。
至于\emphB{非退化性},若范数为零,则所有分量均为零,自然成为零张量;
反之,对于零张量,所有分量为零,范数也为零。
将 $\T{\Phi}$ 乘上 $\lambda$,则有
\begin{align}
	\norm[\Tensors{p}]{\lambda\,\T{\Phi}}
	&=\sqrt{\sum\nolimits_{i_1,\,\cdots,\,i_p=1}^{m}
		\qty\big(\lambda\,
			\Phi\midscript{\orthIdx{i_1,\,\cdots,\,i_p}})^2} \notag \\
	&=\sqrt{\lambda^2 \sum\nolimits_{i_1,\,\cdots,\,i_p=1}^{m}
			\qty\big(\Phi\midscript{\orthIdx{i_1,\,\cdots,\,i_p}})^2}
		\notag \\
	&=\abs{\lambda} \sqrt{\sum\nolimits_{i_1,\,\cdots,\,i_p=1}^{m}
			\qty\big(\Phi\midscript{\orthIdx{i_1,\,\cdots,\,i_p}})^2}
		\notag \\
	&=\abs{\lambda}\cdot\norm[\Tensors{p}]{\T{\Phi}} \comma
\end{align}
于是\emphB{正齐次性}也得以验证。
最后,利用 Cauchy--Schwarz 不等式,可有
\begin{align}
	&\alspace\norm[\Tensors{p}]{\T{\Phi}+\T{\Psi}}^2 \notag \\
	&=\sum \qty\Big(\Phi\midscript{\orthIdx{i_1,\,\cdots,\,i_p}}
			+\Psi\midscript{\orthIdx{i_1,\,\cdots,\,i_p}})^2 \notag \\
	&=\sum \qty[
			\qty\big(\Phi\midscript{\orthIdx{i_1,\,\cdots,\,i_p}})^2
			+2\, \Phi\midscript{\orthIdx{i_1,\,\cdots,\,i_p}} \,
				\Psi\midscript{\orthIdx{i_1,\,\cdots,\,i_p}}
			+\qty\big(\Psi\midscript{\orthIdx{i_1,\,\cdots,\,i_p}})^2]
		\notag \\
	&=\sum \qty\big(\Phi\midscript{\orthIdx{i_1,\,\cdots,\,i_p}})^2
		+2\sum \Phi\midscript{\orthIdx{i_1,\,\cdots,\,i_p}} \,
			\Psi\midscript{\orthIdx{i_1,\,\cdots,\,i_p}}
		+\sum \qty\big(\Psi\midscript{\orthIdx{i_1,\,\cdots,\,i_p}})^2
		\notag \\
	&\leqslant\norm[\Tensors{p}]{\T{\Phi}}^2
		+2 \sqrt{\sum \qty\big(
				\Phi\midscript{\orthIdx{i_1,\,\cdots,\,i_p}})^2}
			\sqrt{\sum \qty\big(
				\Psi\midscript{\orthIdx{i_1,\,\cdots,\,i_p}})^2}
		+\norm[\Tensors{p}]{\T{\Phi}}^2 \notag \\
	&=\norm[\Tensors{p}]{\T{\Phi}}^2
		+2\norm[\Tensors{p}]{\T{\Phi}}\cdot\norm[\Tensors{p}]{\T{\Psi}}
		+\norm[\Tensors{p}]{\T{\Phi}}^2 \notag \\
	&=\qty\Big(\norm[\Tensors{p}]{\T{\Phi}}
		+\norm[\Tensors{p}]{\T{\Phi}})^2 \fullstop
\end{align}
两边开方,即为\emphB{三角不等式}。

\blankline

由此,我们就完整地给出了张量大小的刻画手段。
可以看出,它实际上就是 Euclid 空间中向量模的直接推广。

\subsection{简单张量的范数}
根据 \ref{subsec:张量的表示与简单张量}~小节中的定义,
简单张量是形如 $\V{\xi}\tp\V{\eta}\tp\V{\zeta}$ 的张量,
其中的 $\V{\xi},\,\V{\eta},\,\V{\zeta}\in\Rm$,
它是三个向量的张量积。
简单张量的范数为
\begin{equation}
	\norm[\Tensors{3}]{\V{\xi}\tp\V{\eta}\tp\V{\zeta}}
	=\norm{\V{\xi}}\cdot\norm{\V{\eta}}\cdot\norm{\V{\zeta}}
	\fullstop \label{eq:简单张量的范数}
\end{equation}

\begin{myProof}
$\V{\xi}\tp\V{\eta}\tp\V{\zeta}$ 的逆变分量为
\begin{equation}
	\qty(\V{\xi}\tp\V{\eta}\tp\V{\zeta})^{ijk}
	\defeq \V{\xi}\tp\V{\eta}\tp\V{\zeta}
		\qty(\V{g}^i,\,\V{g}^j,\,\V{g}^k)
	=\xi^i \eta^j \zeta^k \fullstop
\end{equation}
同理,它的协变分量为
\begin{equation}
	\qty(\V{\xi}\tp\V{\eta}\tp\V{\zeta})_{ijk}
	\defeq \V{\xi}\tp\V{\eta}\tp\V{\zeta}
		\qty(\V{g}_i,\,\V{g}_j,\,\V{g}_k)
	=\xi_i \eta_j \zeta_k \fullstop
\end{equation}
二者相乘,有
\begin{align}
	&\alspace\qty(\V{\xi}\tp\V{\eta}\tp\V{\zeta})^{ijk}
		\cdot \qty(\V{\xi}\tp\V{\eta}\tp\V{\zeta})_{ijk} \notag \\
	&=\qty(\xi^i \eta^j \zeta^k)
		\cdot \qty(\xi_i \eta_j \zeta_k) \notag \\
	&=\qty(\xi^i \xi_i) \cdot \qty(\eta^j \eta_j)
		\cdot \qty(\zeta^k \zeta_k) \fullstop
\end{align}
注意到
\begin{align}
	\norm{\xi}^2
	&=\ipb{\xi}{\xi} \notag
	\intertext{分别把二者用协变和逆变分量表示:}
	&=\ipb{\xi^i\,\V{g}_i}{\xi_j\,\V{g}^j} \notag \\
	&=\xi^i \xi_j \ipb{\V{g}_i}{\V{g}^j} \notag \\
	&=\xi^i \xi_j \KroneckerDelta{j}{i}
	=\xi^i \xi_i \comma
\end{align}
于是
\begin{equation}
	\qty(\V{\xi}\tp\V{\eta}\tp\V{\zeta})^{ijk}
		\cdot \qty(\V{\xi}\tp\V{\eta}\tp\V{\zeta})_{ijk}
	=\norm{\xi}^2 \cdot \norm{\V{\eta}}^2 \cdot \norm{\V{\zeta}}^2
	\fullstop
\end{equation}
两边开方,即得 \eqref{eq:简单张量的范数}~式。
\end{myProof}

\section{张量场沿坐标曲线的变化率}
在区域 $\domD{\V{x}}\subset\Rm$ 上,
若存在一个自变量用\emphB{位置}刻画的映照
\begin{equation}
	\mmap{\T{\Phi}}{\domD{\V{x}}\ni\V{x}}
		{\T{\Phi}(\V{x})\in\Tensors{r}} \comma
\end{equation}
则称张量 $\T{\Phi}(\V{x})$ \footnote{%
	类似“$\T{\Phi}(\V{x})$”的记号在前文也表示张量 $\T{\Phi}$
	\emphB{作用}在向量 $\V{x}$ 上(“吃掉”了 $\V{x}$),此时有
	$\T{\Phi}(\V{x})\in\realR$,注意不要混淆。
	符号有限,难免如此,还望诸位体谅。}是定义在 $\domD{\V{x}}$
上的一个\emphA{张量场}。

下面我们以三阶张量为例。设在物理域 $\domD{\V{X}}\subset\Rm$
和参数域 $\domD{\V{x}}\subset\Rm$ 之间已经建立了微分同胚
$\V{X}(\V{x})\in\cf{\domD{\V{x}}}{\domD{\V{X}}}$。
在 $\V{X}(\V{x})$ 处,张量场 $\T{\Phi}(\V{x})$
可以用分量形式表示为
\begin{equation}
	\T{\Phi}(\V{x})=\tensor{\Phi(\V{x})}{^i_j^k}\,
		\V{g}_i(\V{x})\tp\V{g}^j(\V{x})\tp\V{g}_k(\V{x})
	\in\Tensors{3} \comma
\end{equation}
其中的 $\V{g}_i(\V{x}),\,\V{g}^j(\V{x}),\,\V{g}_k(\V{x})$
都是\emphB{局部}基,而张量分量则定义为\footnote{%
	请注意,下式 $\T{\Phi}$ 之后的第一个圆括号表示\emphB{位于}
	$\V{x}$ 处;而后面的方括号则表示\emphB{作用在}这几个向量上。}
\begin{equation}
	\tensor{\Phi(\V{x})}{^i_j^k}
	\defeq \T{\Phi}(\V{x})
		\qty[\V{g}_i(\V{x}),\,\V{g}^j(\V{x}),\,\V{g}_k(\V{x})]
	\in\realR \fullstop
\end{equation}
类似地,当点沿着 $x^\mu$-线运动到
$\V{X}\qty(\V{x}+\lambda\,\V{e}_\mu)$ 处时,有
\begin{equation}
	\T{\Phi}\qty(\V{x}+\lambda\,\V{e}_\mu)
	=\tensor{\Phi\qty(\V{x}+\lambda\,\V{e}_\mu)}{^i_j^k}\,
		\V{g}_i \qty(\V{x}+\lambda\,\V{e}_\mu)
		\tp\V{g}^j \qty(\V{x}+\lambda\,\V{e}_\mu)
		\tp\V{g}_k \qty(\V{x}+\lambda\,\V{e}_\mu) \fullstop
\end{equation}

现在研究 $\lambda\to 0 \in\realR$ 时的极限
\begin{equation}
	\lim_{\lambda\to 0}
	\frac{\T{\Phi}\qty(\V{x}+\lambda\,\V{e}_\mu)-\T{\Phi}(\V{x})}
		{\lambda}
	\eqcolon \pdv{\T{\Phi}}{x^\mu} (\V{x})
	\in\Tensors{3} \fullstop
\end{equation}