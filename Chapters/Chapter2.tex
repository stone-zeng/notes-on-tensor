\section{张量积}
	\emphA{张量积}也叫\emphA{张量并},用符号“$\tp$”表示。
	在 \ref{subsec:张量的表示与简单张量}~小节给出简单张量的定义时,
	实际上就用到了张量积。张量积的定义为:
	\begin{align}
		\forall\,\T{\Phi}\in\Tensors{p},\,\T{\Psi}\in\Tensors{q},
		&\mathrel{\phantom{=}} \T{\Phi}\tp\T{\Psi}
			\in\Tensors{p+q} \notag \\
		&=\qty(\Phi^{i_1 \cdots i_p} \,
				\V{g}_{i_1}\tp\cdots\tp\V{g}_{i_p})
			\tp \qty(\Psi_{j_1 \cdots j_q} \,
				\V{g}^{j_1}\tp\cdots\tp\V{g}^{j_q}) \notag \\
		&\defeq \Phi^{i_1 \cdots i_p} \,
			\Psi_{j_1 \cdots j_q}\,
			\qty(\V{g}_{i_1}\tp\cdots\tp\V{g}_{i_p})
			\tp \qty(\V{g}^{j_1}\tp\cdots
				\tp\V{g}^{j_q}_{\phantom{i_p}}) \fullstop
	\end{align}
	由该定义可以知道,关于简单张量 $\qty(\V{g}_{i_1}\tp\cdots
		\tp\V{g}_{i_p}) \tp \qty(\V{g}^{j_1}\tp\cdots
		\tp\V{g}^{j_q}_{\phantom{i_p}})$,相应的张量分量为
	\begin{equation}
		\tensor{\qty\big(\Phi\tp\Psi)}
			{^{i_1 \cdots i_p}_{j_1 \cdots j_q}} \fullstop
	\end{equation}
	
\section{\texorpdfstring{$e$ 点积}{e 点积}}
	张量的 \emphA{$e$ 点积}可以用符号“$\edp$”表示。
	从这个符号可以看出 $e$ 点积的作用:前 $e$ 个指标缩并,后面的点乘。
	
	对于任意的 $\T{\Phi}\in\Tensors{p},\,
		\T{\Psi}\in\Tensors{q},\,
		e\leqslant\min\qty{p,\,q}\in\natN$,$e$ 点积是这样定义的:
	\begin{align}
		&\mathrel{\phantom{=}} \T{\Phi}\edp\T{\Psi} \notag \\
		&=\qty(\Phi^{i_1 \cdots i_{p-e} i_{p-e+1} \cdots i_p} \,
			\V{g}_{i_1}\tp\cdots\tp\V{g}_{i_{p-e}}
			\tp\hl{\V{g}_{i_{p-e+1}}\tp\cdots\tp\V{g}_{i_p}}
			) \notag \\
		&\mathrel{\phantom{=}}\quad\edp
			\qty(\Psi^{j_1 \cdots j_e j_{e+1} \cdots j_q} \,
			\hl{\V{g}_{j_1}\tp\cdots\tp\V{g}_{j_e}}
			\tp\V{g}_{j_{e+1}}\tp\cdots\tp\V{g}_{j_q}) \notag
		\intertext{把高亮的部分做内积,得到\emphB{度量}:}
		&\defeq\Phi^{i_1 \cdots i_{p-e} i_{p-e+1} \cdots i_p} \,
			\Psi^{j_1 \cdots j_e j_{e+1} \cdots j_q} \notag \\
		&\mathrel{\phantom{=}}\quad\cdotp
			g_{i_{p-e+1} j_1} \cdots g_{i_p j_e} \,
			\qty(\V{g}_{i_1}\tp\cdots\tp\V{g}_{i_{p-e}})
			\tp\qty(\V{g}_{j_{e+1}}\tp\cdots\tp\V{g}_{j_q}) \notag
		\intertext{玩一下“指标升降游戏”(注意有两种结合方式:
			与 $\Phi$ 或 $\Psi$),可得}
		&=\left\{\begin{lgathered}
				\tensor{\Phi}{^{i_1 \cdots i_{p-e}}_{\hl{j_1 \cdots j_e}}} \,
				\Psi^{\hl{j_1 \cdots j_e} j_{e+1} \cdots j_q} \\
				\Phi^{i_1 \cdots i_{p-e} \hl{i_{p-e+1} \cdots i_p}} \,
				\tensor{\Psi}{_{\hl{i_{p-e+1} \cdots i_p}}^{j_{e+1}
					\cdots j_q}}
			\end{lgathered}\right\}
			\qty(\V{g}_{i_1}\tp\cdots\tp\V{g}_{i_{p-e}})
			\tp\qty(\V{g}_{j_{e+1}}\tp\cdots\tp\V{g}_{j_q}) \fullstop
	\end{align}
	最后一步的大括号中,高亮的 $j_1 \cdots j_e$
	和 $i_{p-e+1} \cdots i_p$ 都是哑标,可以通过求和求掉。因此有
	\begin{equation}
		\T{\Phi}\edp\T{\Psi} \in \Tensors{p+q-2e} \fullstop
	\end{equation}
	换句话说,$e$ 点积的作用就是将指标\emphB{哑标化}。
	
	作为一个特殊的应用,接下来我们介绍\emphA{全点积},用符号“\fdp”表示。
	对于任意的 $\T{\Phi},\,\T{\Psi}\in\Tensors{p}$,有
	\begin{align}
		&\mathrel{\phantom{=}} \T{\Phi}\fdp\T{\Psi}
			\defeq \T{\Phi}\edp[p]\T{\Psi} \notag \\
		&=\qty(\Phi^{i_1 \cdots i_p}\,\V{g}_{i_1}\tp\cdots\tp\V{g}_{i_p})
			\edp[p]
			\qty(\Psi^{j_1 \cdots j_p}\,\V{g}_{j_1}\tp\cdots\tp\V{g}_{j_p})
			\notag \\
		&=\Phi^{i_1 \cdots i_p} \, \Psi^{j_1 \cdots j_p} \,
			g_{i_1 j_1} \cdots g_{i_p j_p} \notag \\
		&=\left\{\begin{lgathered}
				\Phi_{j_1 \cdots j_p} \, \Psi^{j_1 \cdots j_p} \\
				\Phi^{i_1 \cdots i_p} \, \Psi_{i_1 \cdots i_p}
			\end{lgathered}\right.
			\in\realR \fullstop
	\end{align}
	可见,全点积将\emphB{全部}指标哑标化。
	
	张量自身和自身的全点积,定义为它的\emphA{范数}:
	\begin{equation}
		\T{\Phi}\fdp\T{\Phi}
		=\Phi^{i_1 \cdots i_p} \, \Phi_{i_1 \cdots i_p}
		\eqcolon \qty|\T{\Phi}|^2_{\Tensors{p}} \fullstop
	\end{equation}
	
\section{叉乘}
	张量的\emphA{叉乘}要求底空间为 $\realR^3$。
	对于任意的 $\T{\Phi}\in\Tensors[\realR^3]{p},\,
	\T{\Psi}\in\Tensors[\realR^3]{q}$,叉乘的定义如下:
	\begin{align}
		&\mathrel{\phantom{=}} \T{\Phi}\cp\T{\Psi} \notag \\
		&=\qty(\Phi^{i_1 \cdots i_{p-1} i_p} \,
				\V{g}_{i_1}\tp\cdots\tp\V{g}_{i_{p-1}}\tp\V{g}_{i_p})
			\cp \qty(\Psi_{j_1 j_2 \cdots j_q} \,
				\V{g}^{j_1}\tp\V{g}^{j_2}\cdots\tp\V{g}^{j_q}) \notag \\
		&\defeq \Phi^{i_1 \cdots i_p} \, \Psi_{j_1 \cdots j_p} \,
			\V{g}_{i_1}\tp\cdots\tp\V{g}_{i_{p-1}}
			\tp\qty(\V{g}_{i_p}\cp\V{g}^{j_1})
			\tp\V{g}^{j_2}\cdots\tp\V{g}^{j_q}
			\in\Tensors[\realR^3]{p+q-1} \fullstop
	\end{align}
	注意到,此时简单张量的维数已经降了一阶。
	
	利用\emphA{Levi-Civita 记号},可以进一步展开上式。
	\begin{align}
		\V{g}_{i_p}\cp\V{g}^{j_1}
		=\LeviCivita{_{i_p}^{j_1}_s}\,\V{g}^s \comma
	\end{align}
	式中的
	\begin{equation}
		\LeviCivita{_{i_p}^{j_1}_s}
		=\det[\V{g}_{i_p},\,\V{g}^{j_1},\,\V{g}_s] \fullstop
	\end{equation}
	于是
	\begin{equation}
		\T{\Phi}\cp\T{\Psi} \,
		=\LeviCivita{_{i_p}^{j_1}_s}\,
			\Phi^{i_1 \cdots i_p} \, \Psi_{j_1 \cdots j_p}
			\V{g}_{i_1}\tp\cdots\tp\V{g}_{i_{p-1}} \tp\V{g}^s
			\tp\V{g}^{j_2}\cdots\tp\V{g}^{j_q} \fullstop
	\end{equation}
	
	下面我们再来类比地定义一种混合积“$\edp[\cp]$”。
	对于任意的 $\T{\Phi},\,\T{\Psi}\in\Tensors{3}$,定义
	\begin{align}
		\T{\Phi}\edp[\cp]\T{\Psi}
		&=\qty(\Phi^{ijk}\,\V{g}_i\tp\V{g}_j\tp\V{g}_k)
			\edp[\cp]\qty(\Psi_{pqr}\,\V{g}^p\tp\V{g}^q\tp\V{g}^r)\notag \\
		&\defeq \Phi^{ijk}\,\Psi_{pqr} \,
			\KroneckerDelta{q}{j} \,
			\V{g}_i\tp\qty(\V{g}_k\cp\V{g}^p)\tp\V{g}^r \notag
		\intertext{缩并掉 Kronecker δ,
			同时利用 Levi-Civita 记号展开叉乘项,可有}
		&=\LeviCivita{_k^p_s}\,\Phi^{ijk}\,\Psi_{pjr}\,
			\V{g}_i\tp\V{g}^s\tp\V{g}^r \comma
	\end{align}
	式中的
	\begin{equation}
		\LeviCivita{_k^p_s}=\det[\V{g}_k,\,\V{g}^p,\,\V{g}_s] \fullstop
	\end{equation}
	
	对于这种混合积,并没有一般的约定。不同的研究者往往会采用不同的写法及表示。
	
\section{置换(一)}
	本节主要介绍\emphA{置换运算}的定义及相关概念,
	这将使我们暂时离开张量运算的主线。
	
	置换运算实际上是一种交换位置或者改变次序的运算。
	之后我们还将引入针对张量的\emph{置换算子},它是外积运算和外微分运算的基础。
	这些运算是现代张量分析与微分几何的支柱。
	
\subsection{置换的定义}
	我们从一个例子开始。下面是一个 $2 \times 7$ 的“矩阵”:
	\begin{equation}
		\Perm{\sigma}=\mqty[
			\circNum{1} & \circNum{2} & \circNum{3} & \circNum{4} &
				\circNum{5} & \circNum{6} & \circNum{7} \\
			\circNum{7} & \circNum{4} & \circNum{5} & \circNum{1} &
				\circNum{6} & \circNum{2} & \circNum{3}
		] \fullstop
		\label{eq:置换序号定义}
	\end{equation}
	矩阵里面的每一个数字表示一个位置。可以想象成 7 把椅子,
	先是按第一行的顺序依次排列,再按照第二行的顺序打乱,重新排列。
	于是这就成为一个\emphA{7 阶置换}。这个定义等价于
	\begin{mySubEq}
		\begin{gather}
			\Perm{\sigma}=\mqty*(
				4 & 9 & 2 & 7 & 5 & 8 & 3 \\
				3 & 7 & 5 & 4 & 8 & 9 & 2
			) \comma \label{eq:置换元素表示_数字}
			\intertext{自然也等价于}
			\Perm{\sigma}=\mqty*(
				\spadesuit & \heartsuit & \diamondsuit & \clubsuit &
					\varspadesuit & \varheartsuit & \vardiamondsuit \\
				\vardiamondsuit & \clubsuit & \varspadesuit & \spadesuit &
					\varheartsuit & \heartsuit & \diamondsuit
			) \comma \label{eq:置换元素表示_符号}
		\end{gather}
	\end{mySubEq}
	当然,换用任何元素也都是可以的。
	
	通常我们用方括号表示置换的\emphA{序号定义},即标号的排列轮换;
	用圆括号表示\emphA{元素定义},即标号对应元素的轮换。
	
\subsection{置换的符号}
	接着来定义置换的\emphA{符号} $\sgn\Perm{\sigma}$。
	这里我们把每次交换两个数字称为一次“操作”。
	如果经过\emphB{偶数次}“操作”,可以把经置换后的序列恢复为原来的顺序,
	那么该置换的符号 $\sgn\Perm{\sigma} = 1$;
	而如果经过\emphB{奇数次}“操作”才可以复原,则 $\sgn\Perm{\sigma}=-1$。
	若用一个式子表示,则为
	\begin{equation}
		\sgn\Perm{\sigma} = (-1)^n \comma
	\end{equation}
	其中的 $n$ 是恢复原本顺序所需“操作”的次数.
	
	下面我们以式~\eqref{eq:置换序号定义} 所定义的 $\Perm{\sigma}$ 为例,
	演示求置换符号的过程。这里的关键是通过两两交换,
	按如下步骤把式~\eqref{eq:置换元素表示_符号} 的第二行变换成第一行:
	\begin{gather*}
		\mqty{
			\hl{\vardiamondsuit} & \hlw{\clubsuit} & \hlw{\varspadesuit} &
				\hl{\spadesuit} & \hlw{\varheartsuit} & \hlw{\heartsuit} &
				\hlw{\diamondsuit}
		} \\
		\mqty{ & & & \Downarrow & & & } \\
		\mqty{
			\hl[pink]{\spadesuit} & \hl{\clubsuit} & \hlw{\varspadesuit} &
				\hl[pink]{\vardiamondsuit} & \hlw{\varheartsuit} &
				\hl{\heartsuit} & \hlw{\diamondsuit}
		} \\
		\mqty{ & & & \Downarrow & & & } \\
		\mqty{
			\hlw{\spadesuit} & \hl[pink]{\heartsuit} & \hl{\varspadesuit} &
				\hlw{\vardiamondsuit} & \hlw{\varheartsuit} &
				\hl[pink]{\clubsuit} & \hl{\diamondsuit}
		} \\
		\mqty{ & & & \Downarrow & & & } \\
		\mqty{
			\hlw{\spadesuit} & \hlw{\heartsuit} & \hl[pink]{\diamondsuit} &
				\hl{\vardiamondsuit} & \hlw{\varheartsuit} & \hl{\clubsuit} &
				\hl[pink]{\varspadesuit}
		} \\
		\mqty{ & & & \Downarrow & & & } \\
		\mqty{
			\hlw{\spadesuit} & \hlw{\heartsuit} & \hlw{\diamondsuit} &
				\hl[pink]{\clubsuit} & \hl{\varheartsuit} &
				\hl[pink]{\vardiamondsuit} & \hl{\varspadesuit}
		} \\
		\mqty{ & & & \Downarrow & & & } \\
		\phantom{\mspace{10mu}}\mqty{
			\hlw{\spadesuit} & \hlw{\heartsuit} & \hlw{\diamondsuit} &
				\hlw{\clubsuit} & \hl[pink]{\varspadesuit} &
				\hl{\vardiamondsuit} &
				\hl{\hl[pink]{\varheartsuit}}
		} \\
		\mqty{ & & & \Downarrow & & & } \\
		\mqty{
			\hlw{\spadesuit} & \hlw{\heartsuit} & \hlw{\diamondsuit} &
				\hlw{\clubsuit} & \hlw{\varspadesuit} &
				\hl[pink]{\varheartsuit} & \hl[pink]{\vardiamondsuit}
		}
	\end{gather*}
	一共进行了 6 次两两交换,因此 $\sgn\Perm{\sigma}=1$。
	
\subsection{置换的复合}
	再定义一个置换
	\begin{equation}
		\Perm{\tau}=\mqty[
			1 & 2 & 3 & 4 & 5 & 6 & 7 \\
			5 & 1 & 7 & 3 & 6 & 4 & 2
		] \fullstop
	\end{equation}
	注意这里用了方括号,因此它是一个\emphB{序号定义}。
	方便起见,以后的序号我们都只用不带圈的普通数字表示。
	考虑之前定义的置换
	\begin{equation}
		\Perm{\sigma}=\mqty[
			1 & 2 & 3 & 4 & 5 & 6 & 7 \\
			7 & 4 & 5 & 1 & 6 & 2 & 3
		] \comma
	\end{equation}
	则 $\Perm{\tau}$ 与 $\Perm{\sigma}$ 的复合
	\begin{equation}
		\Perm{\tau}\comp\Perm{\sigma}=
		\qty(\begin{array}{@{}ccccccc@{}}
			\dicei & \diceii & \diceiii & \diceiv & \dicev &
				\dicevi & \circledtwodots \\
			\circledtwodots & \diceiv & \dicev & \dicei & \dicevi &
				\diceii & \diceiii \\
			\hdashline
			\dicevi & \circledtwodots & \diceiii & \dicev & \diceii &
				\dicei & \diceiv
		\end{array})
		\quad\mqty{\\ \leftarrow\Perm{\sigma} \\ \leftarrow\Perm{\tau}}
	\end{equation}
	与函数、线性变换等的复合类似,这里也用小圆圈“$\comp$”表示置换的复合。
	
	假设经过置换 $\Perm{\sigma}$、$\Perm{\tau}$ 作用后得到的序列,
	分别需要 $p$ 次和 $q$ 次两两交换才能复原为原来的序列。
	那么很显然,经过复合置换 $\Perm{\tau}\comp\Perm{\sigma}$ 作用后的序列,
	经过 $q+p$ 次两两交换也一定可以复原。因此,复合置换的符号
	\begin{equation}
		\sgn\qty(\Perm{\tau}\comp\Perm{\sigma})
		=(-1)^{q+p}=(-1)^q \cdotp (-1)^p
		=\sgn\Perm{\tau}\cdotp\sgn\Perm{\sigma} \fullstop
	\end{equation}
	
\subsection{逆置换}
	逆置换 $\Perm{\sigma}^{-1}$ 的定义为
	\begin{equation}
		\Perm{\sigma}^{-1}\comp\Perm{\sigma} = \Id \comma
	\end{equation}
	其中的“$\Id$”是\emphA{恒等映照}。
	
	仍然使用式~\eqref{eq:置换元素表示_符号}:
	\begin{equation}
		\Perm{\sigma}=\mqty*(
			\spadesuit & \heartsuit & \diamondsuit & \clubsuit &
				\varspadesuit & \varheartsuit & \vardiamondsuit \\
			\vardiamondsuit & \clubsuit & \varspadesuit & \spadesuit &
				\varheartsuit & \heartsuit & \diamondsuit
		) \comma
	\end{equation}
	那么自然有
	\begin{equation}
		\Perm{\sigma}^{-1}=\mqty*(
			\vardiamondsuit & \clubsuit & \varspadesuit & \spadesuit &
				\varheartsuit & \heartsuit & \diamondsuit \\
			\spadesuit & \heartsuit & \diamondsuit & \clubsuit &
				\varspadesuit & \varheartsuit & \vardiamondsuit
		) \fullstop
	\end{equation}
	显然,我们有 $\Perm{\sigma}^{-1}\comp\Perm{\sigma} = \Id$。
	
	回忆一下逆矩阵的定义。矩阵 $\Mat{A}$ 的逆 $\Mat{A}^{-1}$ 既要满足
	$\Mat{A}^{-1}\Mat{A}=\Mat{I}$,又要满足
	$\Mat{A}\Mat{A}^{-1}=\Mat{I}$。对于置换也是如此,
	因此我们需要检查 $\Perm{\sigma}\comp\Perm{\sigma}^{-1}$:\footnote{%
		该式中的数字角标用来澄清原始序号。}
	\begin{equation}
		\Perm{\sigma}\comp\Perm{\sigma}^{-1}=
		\qty(\begin{array}{@{}ccccccc@{}}
			\vardiamondsuit & \clubsuit & \varspadesuit & \spadesuit &
				\varheartsuit & \heartsuit & \diamondsuit \\
			\spadesuit_1 & \heartsuit_2 & \diamondsuit_3 & \clubsuit_4 &
				\varspadesuit_5 & \varheartsuit_6 & \vardiamondsuit_7 \\
			\hdashline
			\vardiamondsuit_7 & \clubsuit_4 & \varspadesuit_5 &
				\spadesuit_1 & \varheartsuit_6 &
				\heartsuit_2 & \diamondsuit_3
		\end{array})
		\quad\mqty{
			\\ \leftarrow\Perm{\sigma}^{-1} \\
			\leftarrow\Perm{\sigma}\phantom{^{-1}}
		}
	\end{equation}
	可见的确有 $\Perm{\sigma}\comp\Perm{\sigma}^{-1}=\Id$。
	
	另外,由于恒等映照 $\Id$ 作用后序列不发生变化,
	复原所需的交换次数为 0,因此
	\begin{equation}
		\sgn\Id=(-1)^0=1 \fullstop
	\end{equation}
	而根据定义,
	\begin{equation}
		\Id=\Perm{\sigma}^{-1}\comp\Perm{\sigma} \comma
	\end{equation}
	故有
	\begin{equation}
		\sgn\Perm{\sigma} \cdotp \sgn\Perm{\sigma}^{-1} = 1 \fullstop
	\end{equation}
	由此,可以推知
	\begin{equation}
		\sgn\Perm{\sigma}=\sgn\Perm{\sigma}^{-1} \comma
	\end{equation}
	即置换与它的逆具有\emphB{相同}的符号。
	
\section{置换(二)}
	本节将介绍置换运算的基本性质。
	
\subsection{置换的穷尽}
	先要做一点铺垫。设有序数组
	\begin{equation*}
		\qty{i_1,\,i_2,\,\cdots,\,i_r}
	\end{equation*}
	经置换 $\Perm{\sigma}$ 作用后成为
	\begin{equation*}
		\qty{\Perm{\sigma}(i_1),\,\Perm{\sigma}(i_2),\,
			\cdots,\,\Perm{\sigma}(i_r)} \comma
	\end{equation*}
	则根据之前的元素定义(圆括号),可以把 $\Perm{\sigma}$ 记为
	\begin{equation}
		\Perm{\sigma}=\mqty*(
			i_1 & i_2 & \cdots & i_r \\
			\Perm{\sigma}(i_1) & \Perm{\sigma}(i_2) &
				\cdots & \Perm{\sigma}(i_r)
		)\fullstop
	\end{equation}
	每次置换都将得到一个有序数组。把它们组合到一起,就可以得到集合
	\begin{equation}
		\set[\bigg]
			{\qty(\Perm{\sigma}(i_1),\,\Perm{\sigma}(i_2),\,
				\cdots,\,\Perm{\sigma}(i_r))}
			{\forall\,\Perm{\sigma}\in\Permutations{r}} \fullstop
	\end{equation}
	其中的 $\Permutations{r}$ 表示 $r$ 阶置换的全体。
	根据排列组合原理,$r$ 阶置换的总数等于 $r$ 个元素的\emphB{全排列数}。
	即该集合共有 $r!$ 个元素。
	
	下面我们要证明
	\begin{mySubEq}
		\begin{align}
			&\mathrel{\phantom{=}}\set[\bigg]
				{\qty(\Perm{\sigma}(i_1),\,\Perm{\sigma}(i_2),\,
					\cdots,\,\Perm{\sigma}(i_r))}
				{\forall\,\Perm{\sigma}\in\Permutations{r}} \notag \\
			%
			&=\set[\bigg]
				{\qty(
					\Perm{\tau}\comp\Perm{\sigma}(i_1),\,
					\Perm{\tau}\comp\Perm{\sigma}(i_2),\,\cdots,\,
					\Perm{\tau}\comp\Perm{\sigma}(i_r) )}
				{\forall\,\Perm{\sigma},\,\Perm{\tau}\in\Permutations{r}}
				\label{eq:置换的穷尽_复合1} \\
			&=\set[\bigg]
				{\qty(
					\Perm{\sigma}\comp\Perm{\tau}(i_1),\,
					\Perm{\sigma}\comp\Perm{\tau}(i_2),\,\cdots,\,
					\Perm{\sigma}\comp\Perm{\tau}(i_r) )}
				{\forall\,\Perm{\sigma},\,\Perm{\tau}\in\Permutations{r}}
				\label{eq:置换的穷尽_复合2} \\
			&=\set[\bigg]
				{\qty(
					\Perm{\sigma}^{-1}(i_1),\,
					\Perm{\sigma}^{-1}(i_2),\,\cdots,\,
					\Perm{\sigma}^{-1}(i_r))}
				{\forall\,\Perm{\sigma}\in\Permutations{r}} 
				\label{eq:置换的穷尽_逆} \fullstop
		\end{align}
	\end{mySubEq}
	\colorbox{pink}{这说明置换构成了置换群。?}
	
	\begin{myProof}
		证明的思路是说明集合互相包含。
		
		对于式~\eqref{eq:置换的穷尽_复合1},
		右边的 $\Perm{\tau}\comp\Perm{\sigma}$ 也是一个 $r$ 阶置换,
		自然符合左边集合的定义,因此 $\text{右边}\subset\text{左边}$。
		由于这一步是相当显然的,以下的几个证明我们将略去该步。
		另一方面,左边的 $\Perm{\sigma}$ 可以表示成
		\begin{equation}
			\Perm{\sigma}
			=\Id\comp\Perm{\sigma}
			=\qty(\Perm{\tau}\comp\Perm{\tau}^{-1}) \comp\Perm{\sigma}
			=\Perm{\tau}\comp \qty(\Perm{\tau}^{-1}\comp\Perm{\sigma})
			\comma
		\end{equation}
		这就是右边集合的定义,因此 $\text{左边}\subset\text{右边}$。
		故可证得等式成立。
		
		对于式~\eqref{eq:置换的穷尽_复合2},我们有
		\begin{equation}
			\Perm{\sigma}
			=\Perm{\sigma}\comp\Id
			=\Perm{\sigma}\comp \qty(\Perm{\tau}^{-1}\comp\Perm{\tau})
			=\qty(\Perm{\sigma}\comp\Perm{\tau}^{-1}) \comp\Perm{\tau}
			\comma
		\end{equation}
		它符合了右边集合的定义,因此 $\text{左边}\subset\text{右边}$。
		于是等式成立。
		
		对于式~\eqref{eq:置换的穷尽_逆},我们有
		\begin{equation}
			\Perm{\sigma}=\qty(\Perm{\sigma}^{-1})^{-1} \comma
		\end{equation}
		它符合了右边集合的定义,因此 $\text{左边}\subset\text{右边}$。
		于是等式成立。
	\end{myProof}
	
\subsection{数组元素的乘积} \label{subsec:数组元素的乘积}
	设有序数组 $\qty{i_1,\,i_2,\,\cdots,\,i_r}$、
	$\qty{j_1,\,j_2,\,\cdots,\,j_r}$ 和 $\qty{k_1,\,k_2,\,\cdots,\,k_r}$
	经 $r$ 阶置换 $\Perm{\sigma}$ 作用后分别成为
	$\qty{\Perm{\sigma}(i_1),\,\Perm{\sigma}(i_2),\,
		\cdots,\,\Perm{\sigma}(i_r)}$、
	$\qty{\Perm{\sigma}(j_1),\,\Perm{\sigma}(j_2),\,
		\cdots,\,\Perm{\sigma}(j_r)}$ 和
	$\qty{\Perm{\sigma}(k_1),\,\Perm{\sigma}(k_2),\,
		\cdots,\,\Perm{\sigma}(k_r)}$,也就是说
	\begin{equation}
		\Perm{\sigma}=\mqty*(
			i_1 & i_2 & \cdots & i_r \\
			\Perm{\sigma}(i_1) & \Perm{\sigma}(i_2) &
				\cdots & \Perm{\sigma}(i_r) )
		=\mqty*(
			j_1 & j_2 & \cdots & j_r \\
			\Perm{\sigma}(j_1) & \Perm{\sigma}(j_2) &
				\cdots & \Perm{\sigma}(j_r) )
		=\mqty*(
			k_1 & k_2 & \cdots & k_r \\
			\Perm{\sigma}(k_1) & \Perm{\sigma}(k_2) &
				\cdots & \Perm{\sigma}(k_r) ) \fullstop
	\end{equation}
	我们有如下结论:
	\begin{equation}
		\forall\,\Perm{\sigma}\in\Permutations{r}\, ,\quad
		A_{i_1 j_1 k_1} A_{i_2 j_2 k_2} \cdots
			A_{i_r j_r k_r}
		=A_{\Perm{\sigma}(i_1)\,\Perm{\sigma}(j_1)\,\Perm{\sigma}(k_1)}
			A_{\Perm{\sigma}(i_2)\,\Perm{\sigma}(j_2)\,\Perm{\sigma}(k_2)}
			\cdots
			A_{\Perm{\sigma}(i_r)\,\Perm{\sigma}(j_r)\,\Perm{\sigma}(k_r)}
			\comma
	\end{equation}
	式中的 $A_{ijk}$ 表示三维数组 $\Mat{A}$ 的一个元素,其指标为 $ijk$。
	
	我们通过一个例子来说明这一条性质。还是用式~\eqref{eq:置换元素表示_数字} 和
	\eqref{eq:置换元素表示_符号} 所定义的置换 $\Perm{\sigma}$:
	\begin{equation}
		\Perm{\sigma}=\mqty*(
			4 & 9 & 2 & 7 & 5 & 8 & 3 \\
			3 & 7 & 5 & 4 & 8 & 9 & 2 )
		=\mqty*(
			\spadesuit & \heartsuit & \diamondsuit & \clubsuit &
				\varspadesuit & \varheartsuit & \vardiamondsuit \\
			\vardiamondsuit & \clubsuit & \varspadesuit & \spadesuit &
				\varheartsuit & \heartsuit & \diamondsuit ) \fullstop
				\label{eq:置换元素表示_数组元素的乘积举例}
	\end{equation}
	随意写出一个数组元素乘积:
	\begin{equation}
		A_{379}A_{264}A_{157}A_{483}A_{698}
		A_{\diamondsuit\clubsuit\heartsuit}
		A_{\vardiamondsuit\varspadesuit\varheartsuit} \fullstop
		\label{eq:数组元素乘积举例}
	\end{equation}
	三组下标分别为
	\begin{equation}
		\left\{\begin{lgathered}
			3,\,2,\,1,\,4,\,6,\,\diamondsuit,\,\vardiamondsuit;\\
			7,\,6,\,5,\,8,\,9,\,\clubsuit,\,\varspadesuit;\\
			9,\,4,\,7,\,3,\,8,\,\heartsuit,\,\varheartsuit.\\
		\end{lgathered}
		\right.
	\end{equation}
	考虑 $\Perm{\sigma}$ 的\emphB{序号定义}式~\eqref{eq:置换序号定义}:
	\begin{equation}
		\Perm{\sigma}=\mqty[
			1 & 2 & 3 & 4 & 5 & 6 & 7 \\
			7 & 4 & 5 & 1 & 6 & 2 & 3
		] \fullstop
	\end{equation}
	所谓序号只是位置的抽象表示,而不代表任何真实的元素。
	请记住:置换始终是\emphB{位置}的变换,而非\emphB{元素}的变换,
	不要被式~\eqref{eq:置换元素表示_数组元素的乘积举例} 给迷惑了。
	把 $\Perm{\sigma}$ 作用在这三组下标上,可得
	\begin{equation}
		\left\{\begin{lgathered}
			\vardiamondsuit,\,4,\,6,\,3,\,\diamondsuit,\,2,\,1;\\
			\varspadesuit,\,8,\,9,\,7,\,\clubsuit,\,6,\,5;\\
			\varheartsuit,\,3,\,8,\,9,\,\heartsuit,\,4,\,7.\\
		\end{lgathered}
		\right.
	\end{equation}
	于是之前的数组元素乘积就变成了
	\begin{equation}
		A_{\vardiamondsuit\varspadesuit\varheartsuit}
		A_{483}A_{698}A_{379}
		A_{\diamondsuit\clubsuit\heartsuit}
		A_{264}A_{157} \fullstop
	\end{equation}
	比对一下各元素,可见与式~\eqref{eq:数组元素乘积举例} 的确是完全一样的。
	
\subsection{哑标的穷尽}
	考虑如下集合:
	\begin{equation}
		\set[\bigg]
		{\qty(i_1,\,i_2,\,\cdots,\,i_r)}
		{\qty{i_1,\,i_2,\,\cdots,\,i_r}
			\text{\ 可取\ } 1,\,2,\,\cdots,\,m}
		\fullstop
	\end{equation}
	每个 $i_k$ 都有 $m$ 种取法,而 $i_k$ 又有 $r$ 个,
	因此该集合一共有 $m^r$ 元素。我们有
	\begin{mySubEq}
		\begin{align}
		\forall\,\Perm{\sigma}\in\Permutations{r}\, ,
			&\mathrel{\phantom{=}}\set[\bigg]
			{\qty(i_1,\,i_2,\,\cdots,\,i_r)}
			{\qty{i_1,\,i_2,\,\cdots,\,i_r}
				\text{\ 可取\ } 1,\,2,\,\cdots,\,m} \notag \\
			&=\set[\bigg]
			{\qty(\Perm{\sigma}(i_1),\,\Perm{\sigma}(i_2),\,
				\cdots,\,\Perm{\sigma}(i_r) )}
			{\qty{i_1,\,i_2,\,\cdots,\,i_r}
				\text{\ 可取\ } 1,\,2,\,\cdots,\,m}
			\label{eq:哑标的穷尽_置换} \\
			&=\set[\bigg]
			{\qty(\Perm{\sigma}^{-1}(i_1),\,\Perm{\sigma}^{-1}(i_2),\,
				\cdots,\,\Perm{\sigma}^{-1}(i_r) )}
			{\qty{i_1,\,i_2,\,\cdots,\,i_r}
				\text{\ 可取\ } 1,\,2,\,\cdots,\,m} \fullstop
			\label{eq:哑标的穷尽_逆置换}
		\end{align}
	\end{mySubEq}
	这里,$i_k$ 起的就是\emphB{哑标}的作用。
	
	\begin{myProof}
		无论怎样置换,$\Perm{\sigma}(i_k)$ 都是 $1,\,2,\,\cdots,\,m$ 中的数。
		因此,对于 $\forall\,\Perm{\sigma}\in\Permutations{r}$,
		\begin{equation}
			\qty(\Perm{\sigma}(i_1),\,\Perm{\sigma}(i_2),\,
				\cdots,\,\Perm{\sigma}(i_r) )
			\in\set[\bigg]
				{\qty(i_1,\,i_2,\,\cdots,\,i_r)}
				{\qty{i_1,\,i_2,\,\cdots,\,i_r}
					\text{\ 可取\ } 1,\,2,\,\cdots,\,m} \comma
		\end{equation}
		即
		\begin{align}
			&\mathrel{\phantom{\subset}}\set[\bigg]
			{\qty(\Perm{\sigma}(i_1),\,\Perm{\sigma}(i_2),\,
				\cdots,\,\Perm{\sigma}(i_r) )}
			{\qty{i_1,\,i_2,\,\cdots,\,i_r}
				\text{\ 可取\ } 1,\,2,\,\cdots,\,m} \notag \\
			&\subset\set[\bigg]
			{\qty(i_1,\,i_2,\,\cdots,\,i_r)}
			{\qty{i_1,\,i_2,\,\cdots,\,i_r}
				\text{\ 可取\ } 1,\,2,\,\cdots,\,m} \fullstop
		\end{align}
		另一方面,由于 $\Id=\Perm{\sigma}^{-1}\comp\Perm{\sigma}$,即
		\begin{equation}
			\qty(i_1,\,i_2,\,\cdots,\,i_r)
			=\qty(\Perm{\sigma}^{-1}\comp\Perm{\sigma}(i_1),\,
				\Perm{\sigma}^{-1}\comp\Perm{\sigma}(i_2),\,
				\cdots,\,\Perm{\sigma}^{-1}\comp\Perm{\sigma}(i_r) ) \comma
		\end{equation}
		而进行一次逆置换仍然使得元素不离开原有的范围,也就是说
		\begin{equation}
			\qty(i_1,\,i_2,\,\cdots,\,i_r)
			\in\set[\bigg]
				{\qty(\Perm{\sigma}(i_1),\,\Perm{\sigma}(i_2),\,
					\cdots,\,\Perm{\sigma}(i_r) )}
				{\qty{i_1,\,i_2,\,\cdots,\,i_r}
					\text{\ 可取\ } 1,\,2,\,\cdots,\,m} \comma
		\end{equation}
		即
		\begin{align}
			&\mathrel{\phantom{\subset}}\set[\bigg]
			{\qty(i_1,\,i_2,\,\cdots,\,i_r)}
			{\qty{i_1,\,i_2,\,\cdots,\,i_r}
				\text{\ 可取\ } 1,\,2,\,\cdots,\,m} \notag \\
			&\subset\set[\bigg]
			{\qty(\Perm{\sigma}(i_1),\,\Perm{\sigma}(i_2),\,
				\cdots,\,\Perm{\sigma}(i_r) )}
			{\qty{i_1,\,i_2,\,\cdots,\,i_r}
				\text{\ 可取\ } 1,\,2,\,\cdots,\,m} \fullstop
		\end{align}
		两个集合互相包含,也就证得了式~\eqref{eq:哑标的穷尽_置换}。
		
		用相同的方法也可证得关于逆置换的 \eqref{eq:哑标的穷尽_逆置换}~式,
		此处从略。
	\end{myProof}
	
\section{置换(三)}
	本节将给出
	
\section{置换(四)}
	本节将回归张量运算的主线,引入\emphA{置换算子}。
	
\subsection{置换算子;对称张量与反对称张量}
	对于任意的置换 $\Perm{\sigma}\in\Permutations{r}$,定义\emphA{置换算子}
	\begin{equation}
		\mdef{\opPerm}
			{\Tensors{r}\ni\T{\Phi}}
			{\opPerm(\T{\Phi})\in\Tensors{r}} \comma
	\end{equation}
	式中
	\begin{equation}
		\opPerm(\T{\Phi})\qty(\V{u}_1,\,\V{u}_2,\,\cdots,\,\V{u}_r)
		\defeq\T{\Phi}\qty(\V{u}_{\Perm{\sigma}(1)},\,
			\V{u}_{\Perm{\sigma}(2)},\,\cdots,\,
			\V{u}_{\Perm{\sigma}(r)})
		\in\realR \fullstop
	\end{equation}
	这里的“$\cdots\in\realR$”是由于张量的定义:\emphB{多重线性函数}。
	
	如果我们的置换
	\begin{equation}
		\Perm{\sigma}=\mqty*(
			i_1 & i_2 & \cdots & i_r \\
			\Perm{\sigma}(i_1) & \Perm{\sigma}(i_1) & \cdots
				& \Perm{\sigma}(i_1)
		) \comma
	\end{equation}
	那么对应的置换算子将满足
	\begin{equation}
		\opPerm(\T{\Phi})\qty(\V{u}_{i_1},\,\V{u}_{i_2},\,
			\cdots,\,\V{u}_{i_r})
		\defeq\T{\Phi}\qty(\V{u}_{\Perm{\sigma}(i_1)},\,
			\V{u}_{\Perm{\sigma}(i_2)},\,\cdots,\,
			\V{u}_{\Perm{\sigma}(i_r)}) \fullstop
	\end{equation}
	
	\blankline
	
	有了置换算子,我们就可以来定义\emphA{对称张量}和\emphA{反对称张量}。
	对称张量的全体记为 $\Sym$,反对称张量的全体记为 $\Skw$。
	如果以 $\Rm$ 为底空间,
	又分别可以记为 $\SymTensors{r}$ 和 $\SkwTensors{r}$。
	
	对于任意的 $\T{\Phi}\in\Tensors{r}$,如果
	\begin{equation}
		\opPerm(\T{\Phi})=\T{\Phi} \comma
	\end{equation}
	则称 $\T{\Phi}$ 为\emphB{对称张量},
	即 $\T{\Phi}\in\Sym \text{\ 或\ } \SymTensors{r}$;如果
	\begin{equation}
		\opPerm(\T{\Phi})=\sgn\Perm{\sigma}\cdotp\T{\Phi} \comma
	\end{equation}
	则称 $\T{\Phi}$ 为\emphB{反对称张量},
	即 $\T{\Phi}\in\Skw \text{\ 或\ } \SkwTensors{r}$。
	
\subsection{置换算子的表示}
	根据上文给出的定义,我们有
	\begin{equation}
		\opPerm(\T{\Phi})\qty(\V{u}_{i_1},\,\cdots,\,\V{u}_{i_r})
		\defeq\T{\Phi}\qty(\V{u}_{\Perm{\sigma}(i_1)},\,\cdots,\,
			\V{u}_{\Perm{\sigma}(i_r)}) \fullstop
	\end{equation}
	首先回忆一下 \ref{subsec:张量的表示与简单张量}~小节中张量的表示:
	选一组基(协变、逆变均可),然后把张量用这组基表示。于是
	\begin{align}
		&\mathrel{\phantom{=}}\opPerm(\T{\Phi})
			\qty(\V{u}_{i_1},\,\cdots,\,\V{u}_{i_r})
		=\T{\Phi}\qty(\V{u}_{\Perm{\sigma}(i_1)},\,\cdots,\,
			\V{u}_{\Perm{\sigma}(i_r)}) \notag
		\intertext{把向量用协变基表示:}
		&=\T{\Phi}\qty(u^{i_1}_{\Perm{\sigma}(i_1)}\,\V{g}_{i_1},\,
			\cdots,\,u^{i_r}_{\Perm{\sigma}(i_r)}\,\V{g}_{i_r}) \notag
		\intertext{根据张量的线性性,提出系数:}
		&=\T{\Phi}\qty(\V{g}_{i_1},\,\cdots,\,\V{g}_{i_r}) \cdotp
			\qty(u^{i_1}_{\Perm{\sigma}(i_1)} \cdots
				u^{i_r}_{\Perm{\sigma}(i_r)}) \notag
		\intertext{前半部分可以用张量分量表示;
			而后半部分是一组逆变分量,可以写成内积的形式}
		&=\tensor{\Phi}{_{i_1 \cdots i_p}}
			\qty[\ipb{\V{u}_{\Perm{\sigma}(i_1)}}{\V{g}^{i_1}}\cdots
				\ipb{\V{u}_{\Perm{\sigma}(i_r)}}{\V{g}^{i_r}}]
		\addtocounter{equation}{1}
		\tag{\theequation*}
		\label{eq:置换算子的表示_中间步骤}
		\intertext{注意到方括号中的其实是简单张量的定义,这就有}
		&=\tensor{\Phi}{_{i_1 \cdots i_p}}
			\V{g}^{i_1}\tp\cdots\tp\V{g}^{i_r}
			\qty(\V{u}_{\Perm{\sigma}(i_1)},\,\cdots,\,
				\V{u}_{\Perm{\sigma}(i_r)}) \fullstop
		\addtocounter{equation}{-1}
	\end{align}
	最后一步仍然没能回到 $\qty(\V{u}_{i_1},\,\cdots,\,\V{u}_{i_r})$,
	因此以上推导只是简单地展开了 $\T{\Phi}$,并没有获得实质性的结果。
	
	然而,只要稍作改动,情况就会大不相同。
	考虑一下 \ref{subsec:数组元素的乘积}~小节中置换有关数组元素乘积的性质:
	\begin{equation}
		\forall\,\Perm{\tau}\in\Permutations{r},\quad
		A_{i_1 j_1} \cdots A_{i_r j_r}
		=A_{\Perm{\tau}(i_1) \Perm{\tau}(j_1)} \cdots
			A_{\Perm{\tau}(i_r) \Perm{\tau}(j_r)} \comma
		\label{eq:置换算子的表示推导_置换性质}
	\end{equation}
	式中
	\begin{equation}
		\Perm{\tau}
		=\mqty*(
			i_1 & \cdots & i_r \\
			\Perm{\tau}(i_1) & \cdots & \Perm{\tau}(i_r) )
		=\mqty*(
			j_1 & \cdots & j_r \\
			\Perm{\tau}(j_1) & \cdots & \Perm{\tau}(j_r) ) \fullstop
	\end{equation}
	由此可以看出,式~\eqref{eq:置换算子的表示_中间步骤} 中方括号中的部分
	其实是由 $\Perm{\sigma}(i_k)$ 和 $i_k$ 两套指标确定的一组数:
	\begin{equation}
		A_{\Perm{\sigma}(i_k)\,i_k}
		=\ipb{\V{u}_{\Perm{\sigma}(i_k)}}{\V{g}^{i_k}} \semicomma
	\end{equation}
	另一方面,显然有 $\Perm{\sigma}^{-1}\in\Permutations{r}$。于是
	\begin{align}
		&\mathrel{\phantom{=}}\opPerm(\T{\Phi})
			\qty(\V{u}_{i_1},\,\cdots,\,\V{u}_{i_r}) \notag \\
		&=\tensor{\Phi}{_{i_1 \cdots i_r}}
			\qty[\ipb{\V{u}_{\Perm{\sigma}(i_1)}}{\V{g}^{i_1}}\cdots
				\ipb{\V{u}_{\Perm{\sigma}(i_r)}}{\V{g}^{i_r}}] \notag
		\intertext{应用置换的性质 \eqref{eq:置换算子的表示推导_置换性质}~式:}
		&=\tensor{\Phi}{_{i_1 \cdots i_r}}
			\qty[
				\ipb{\V{u}_{\Perm{\sigma}^{-1}\comp\Perm{\sigma}(i_1)}}
					{\V{g}^{\Perm{\sigma}^{-1}(i_1)}} \cdots
				\ipb{\V{u}_{\Perm{\sigma}^{-1}\comp\Perm{\sigma}(i_r)}}
					{\V{g}^{\Perm{\sigma}^{-1}(i_r)}}
			] \notag \\
		&=\tensor{\Phi}{_{i_1 \cdots i_r}}
			\qty[
				\ipb{\V{u}_{i_1}}{\V{g}^{\Perm{\sigma}^{-1}(i_1)}} \cdots
				\ipb{\V{u}_{i_2}}{\V{g}^{\Perm{\sigma}^{-1}(i_r)}}
			] \notag
		\intertext{同样,用简单张量表示,可得}
		&=\tensor{\Phi}{_{i_1 \cdots i_r}}
			\V{g}^{\Perm{\sigma}^{-1}(i_1)}\tp\cdots
				\tp\V{g}^{\Perm{\sigma}^{-1}(i_r)}
			\qty(\V{u}_{i_1},\,\cdots,\,\V{u}_{i_r}) \fullstop
	\end{align}
	这样,我们就得到了置换算子的表示:
	\begin{align}
		\opPerm(\T{\Phi})
		&=\opPerm\qty(\tensor{\Phi}{_{i_1 \cdots i_r}}
			\V{g}^{i_1}\tp\cdots\tp\V{g}^{i_r}) \notag \\
		&=\tensor{\Phi}{_{i_1 \cdots i_r}}
			\V{g}^{\Perm{\sigma}^{-1}(i_1)}\tp\cdots
				\tp\V{g}^{\Perm{\sigma}^{-1}(i_r)} \fullstop
	\end{align}
%		
%	
%	
%	