\section{\texorpdfstring{$\DiffMorp$ 微分同胚}{C\^{}p 微分同胚}}
\subsection{双射}
设 $f$ 是集合 $A$ 到 $B$ 的映射。
如果 $A$ 中不同的元素有不同的像,则称 $f$ 为\emphA{单射}(也叫“一对一”);
如果 $B$ 中每个元素都是 $A$ 中元素的像,则称 $f$ 为\emphA{满射};
如果 $f$ 既是单射又是满射,则称 $f$ 为\emphA{双射}(也叫“一一对应”)。
三种情况的示意见图~\ref{fig:单射满射双射}。

\begin{figure}[h]
	\centering
	\includegraphics{Images/Three_Maps.PNG}
	\caption{单射、满射与双射}
	\label{fig:单射满射双射}
\end{figure}

设开集 $\domD{\V{X}},\,\domD{\V{x}}\in\Rm$,
它们之间存在\emphB{双射},即\emphB{一一对应}关系:
\begin{equation}
	\mmap{\V{X}(\V{x})}
		{\domD{\V{x}}\ni\V{x}=\mqty[x^1 \\ \vdots \\ x^m]}
		{\V{X}(\V{x})=\mqty[X^1 \\ \vdots \\ X^m](\V{x})
			\in\domD{\V{X}}} \fullstop
\end{equation}
由于该映照实现了 $\domD{\V{x}}$ 到 $\domD{\V{X}}$ 之间的双射,
因此它存在逆映照:
\begin{equation}
	\mmap{\V{x}(\V{X})}
		{\domD{\V{X}}\ni\V{X}=\mqty[X^1 \\ \vdots \\ X^m]}
		{\V{x}(\V{X})=\mqty[x^1 \\ \vdots \\ x^m](\V{X})
			\in\domD{\V{x}}} \fullstop
\end{equation}

我们把 $\domD{\V{X}}$ 称为\emphA{物理域},它是实际物理事件发生的区域;
$\domD{\V{x}}$ 则称为\emphA{参数域}。
由于物理域通常较为复杂,因此我们常把参数域取为规整的形状,以便之后的处理。

设物理量 $f(\V{X})$ 定义在物理域
$\domD{\V{X}}\in\realR^3$ 上\footnote{
	实际的物理事件当然发生在三维 Euclid 空间中(只谈“空间”),
	但在数学上也可以推广到 $m$ 维。
},则 $f$ 就定义了一个\emphA{场}:
\begin{equation}
	\mmap{f}
		{\domD{\V{X}}\ni\V{X}}
		{f(\V{X})} \fullstop
\end{equation}
所谓的“场”,就是自变量用\emphB{位置}刻画的映照。
它可以是\emphA{标量场},如温度、压强、密度等,此时 $f(\V{X})\in\realR$;
也可以是\emphA{向量场},如速度、加速度、力等,此时 $f(\V{X})\in\realR^3$;
对于更深入的物理、力学研究,往往还需引入\emphA{张量场},
此时 $f(\V{X})\in\Tensors[\realR^3]{r}$。

$\V{X}$ 存在于物理域 $\domD{\V{X}}$ 中,我们称它为\emphA{物理坐标}。
由于上文已经定义了 $\domD{\V{x}}$ 到 $\domD{\V{X}}$ 之间的双射
(不是 $f$!),因此 $\domD{\V{x}}$ 中就有\emphB{唯一的}
$\V{x}$ 与 $\V{X}$ 相对应,
它称为\emphA{参数坐标}(也叫\emphA{曲线坐标})。

这样一来,由于物理域 $\domD{\V{X}}$ 上已经定义了场 $f(\V{X})$,
参数域中也必然存在\emphB{唯一的}一个场 $\tilde{f}(\V{x})$ 与之对应:
\begin{equation}
	\mmap{\tilde{f}}
		{\domD{\V{x}}\ni\V{x}}
		{\tilde{f}(\V{x})=f\comp\V{X}(\V{x})
			=f\qty\big(\V{X}(\V{x}))} \fullstop
\end{equation}
$\V{x}$ 与 $\V{X}$ 是完全等价的,因而 $\tilde{f}$ 与 $f$ 也是完全等价的。

物理域中的场要满足\emphB{守恒定律},如质量守恒、动量守恒、能量守恒等。
从数学上来看,这些守恒定律就是需要满足的一系列\emphB{偏微分方程}。
将场转化到参数域后,它仍要满足这些方程。
但我们已经设法将参数域取得较为规整,故在其上进行数值求解就会相当方便。

开集:任意一点吹出一个球,这个球上的每个点都落在区域内