\section{微分同胚}
\subsection{双射}
设 $f$ 是集合 $A$ 到 $B$ 的映射。
如果 $A$ 中不同的元素有不同的像,则称 $f$ 为\emphA{单射}(也叫“一对一”);
如果 $B$ 中每个元素都是 $A$ 中元素的像,则称 $f$ 为\emphA{满射};
如果 $f$ 既是单射又是满射,则称 $f$ 为\emphA{双射}(也叫“一一对应”)。
三种情况的示意见图~\ref{fig:单射满射双射}。

\begin{figure}[h]
	\centering
	\includegraphics{Images/Three_Maps.PNG}
	\caption{单射、满射与双射}
	\label{fig:单射满射双射}
\end{figure}

设开集 $\domD{\V{x}},\,\domD{\V{X}}\in\Rm$,
它们之间存在\emphB{双射},即\emphB{一一对应}关系:
\begin{equation}
	\mmap{\V{X}(\V{x})}
		{\domD{\V{x}}\ni\V{x}=\mqty[x^1 \\ \vdots \\ x^m]}
		{\V{X}(\V{x})=\mqty[X^1 \\ \vdots \\ X^m](\V{x})
			\in\domD{\V{X}}} \fullstop
\end{equation}
由于该映照实现了 $\domD{\V{x}}$ 到 $\domD{\V{X}}$ 之间的双射,
因此它存在逆映照:
\begin{equation}
	\mmap{\V{x}(\V{X})}
		{\domD{\V{X}}\ni\V{X}=\mqty[X^1 \\ \vdots \\ X^m]}
		{\V{x}(\V{X})=\mqty[x^1 \\ \vdots \\ x^m](\V{X})
			\in\domD{\V{x}}} \fullstop
\end{equation}

我们把 $\domD{\V{x}}$ 称为\emphA{物理域},它是实际物理事件发生的区域;
$\domD{\V{X}}$ 则称为\emphA{参数域}。
由于物理域通常较为复杂,因此我们常把参数域取为规整的形状,以便之后的处理。

设物理量 $f(\V{x})$ 定义在物理域
$\domD{\V{x}}\in\Rm$ 上\footnote{
	实际的物理事件当然只会发生在三维 Euclid 空间中(只就“空间”而言),
	但在数学上也可以推广到 $m$ 维。
},则 $f$ 就定义了一个\emphA{场}:
\begin{equation}
	\mmap{f}
		{\domD{\V{x}}\ni\V{x}}
		{f(\V{x})} \fullstop
\end{equation}
所谓的“场”,就是自变量用\emphB{位置}刻画的映照。
它可以是\emphA{标量场},如温度、压强、密度等,此时 $f(\V{x})\in\realR$;
也可以是\emphA{向量场},如速度、加速度、力等,此时 $f(\V{x})\in\Rm$;
对于更深入的物理、力学研究,往往还需引入\emphA{张量场},
此时 $f(\V{x})\in\Tensors{r}$。

$\V{x}$ 存在于物理域 $\domD{\V{x}}$ 中,我们称它为\emphA{物理坐标}。
由于上文已经定义了 $\domD{\V{x}}$ 到 $\domD{\V{X}}$ 之间的双射
(不是 $f$!),因此 $\domD{\V{X}}$ 中就有\emphB{唯一的}
$\V{X}$ 与 $\V{x}$ 相对应,
它称为\emphA{参数坐标}(也叫\emphA{曲线坐标})。
又因为物理域 $\domD{\V{x}}$ 上已经定义了场 $f(\V{x})$,
参数域中必然\emphB{唯一}存在场 $\tilde{f}(\V{X})$ 与之对应:
\begin{equation}
	\mmap{\tilde{f}}
		{\domD{\V{X}}\ni\V{X}}
		{\tilde{f}(\V{X})=f\comp\V{x}(\V{X})
			=f\qty\big(\V{x}(\V{X}))} \fullstop
\end{equation}
$\V{X}$ 与 $\V{x}$ 是完全等价的,因而 $\tilde{f}$ 与 $f$ 也是完全等价的,
所以同样有
\begin{equation}
	f(\V{x})=\tilde{f}\qty\big(\V{X}(\V{x})) \fullstop
\end{equation}

物理域中的场要满足\emphB{守恒定律},如质量守恒、动量守恒、能量守恒等。
从数学上看,这些守恒定律就是 $f(\V{x})$ 需要满足的一系列\emphB{偏微分方程}。
将场变换到参数域后,它仍要满足这些方程。
但我们已经设法将参数域取得较为规整,故在其上进行数值求解就会相当方便。

\subsection{参数域方程}
上文已经提到,物理域中的场 $f(\V{x})$ 需满足守恒定律,
这等价于一系列偏微分方程(PDE)。
在物理学和力学中,用到的 PDE 通常是\emphB{二阶}的,它们可以写成
\begin{equation}
	\forall\,\V{x}\in\domD{\V{x}},\quad
	\sum_{\alpha=1}^{m} A_\alpha(\V{x}) \pdv{f}{x^\alpha} (\V{x})
	+\sum_{\alpha=1}^{m}\sum_{\beta=1}^{m}
		B_{\alpha\beta}(\V{x}) \pdv{f}{x^\beta}{x^\alpha} (\V{x}) = 0
\end{equation}
的形式。我们的目标是把该\emphB{物理域}方程转化为\emphB{参数域}方程,
即关于 $\tilde{f}(\V{X})$ 的 PDE。
多元微积分中已经提供了解决方案:\emphA{链式求导法则}。

考虑到
\begin{equation}
	f(\V{x})=\tilde{f}\qty\big(\V{X}(\V{x}))
	=\tilde{f}\qty(X^1(\V{x}),\,\cdots,\,X^m(\V{x})) \comma
\end{equation}
于是有
\begin{equation}
	\pdv{f}{x^\alpha} (\V{x})
	=\sum_{s=1}^{m} \pdv{\tilde{f}}{\displaystyle X^s}
		\qty\big(\V{X}(\V{x})) \cdotp
		\pdv{X^s}{x^\alpha} (\V{x}) \fullstop
	\label{eq:参数域方程_一阶偏导项}
\end{equation}
这里用到的链式法则,由\emphB{复合映照可微性定理}驱动,
它要求 $\tilde{f}$ 关于 $\V{X}$ 可微,同时 $\V{X}$ 关于 $\V{x}$ 可微。

%对于更高阶的项,需要更强的条件。一般地,我们要求
%\begin{braceEq}
%	\V{x}(\V{X})&\in\cf{\domD{\V{X}}}{\Rm} \semicomma \\
%	\V{X}(\V{x})&\in\cf{\domD{\V{x}}}{\Rm} \fullstop
%\end{braceEq}
%这里的 $\DiffMorp$ 指\emphB{直至 $p$ 阶偏导数(存在且)连续}的映照全体;
%当 $p$ 等于 1 时,这就等价于可微。

通常情况下,已知条件所给定的都是 $\domD{\V{X}}$ 到 $\domD{\V{x}}$ 的映射
\begin{equation}
	\mmap{\V{x}(\V{X})}
		{\domD{\V{X}}\ni\V{X}=\mqty[X^1 \\ \vdots \\ X^m]}
		{\V{x}(\V{X})=\mqty[x^1 \\ \vdots \\ x^m](\V{X})
			\in\domD{\V{x}}} \comma
\end{equation}
用它不好直接得到式~\eqref{eq:参数域方程_一阶偏导项} 中的
$\pdv*{X^s}{x^\alpha}$ 项,但获得它的“倒数”
$\pdv*{x^\alpha}{X^s}$ 却很容易,只需利用 \emphA{Jacobi 矩阵}:
\begin{equation}
	\JacobiD{\V{x}}(\V{X})
	\defeq\mqty[
		\displaystyle\pdv{x^1}{\displaystyle X^1} & \cdots &
			\displaystyle\pdv{x^1}{\displaystyle X^m} \\[1ex]
		\vdots & \ddots & \vdots \\[0.5ex]
		\displaystyle\pdv{x^m}{\displaystyle X^1} & \cdots &
			\displaystyle\pdv{x^m}{\displaystyle X^m}
		] (\V{X}) \in\realR^{m \times m} \comma
\end{equation}
它是一个方阵。

有了 Jacobi 矩阵,施加一些手法就可以得到所需要的 $\pdv*{X^s}{x^\alpha}$ 项。
考虑到
\begin{equation}
	\forall\,\V{x}\in\domD{\V{x}},\quad
	\V{x}\qty\big(\V{X}(\V{x}))=\V{x} \comma
\end{equation}
并且其中的 $\V{x}(\V{X})$ 和 $\V{X}(\V{x})$ 均可微,可以得到
\begin{equation}
	\JacobiD{\V{x}}\qty\big(\V{X}(\V{x})) 
	\cdotp \JacobiD{\V{X}}(\V{x})
	=\Mat{I}_{m \times m} \comma
\end{equation}
其中的 $\Mat{I}_{m \times m}$ 是单位阵。因此
\begin{equation}
	\JacobiD{\V{X}}(\V{x})
	\defeq\mqty[
		\displaystyle\pdv{X^1}{\displaystyle x^1} & \cdots &
			\displaystyle\pdv{X^1}{\displaystyle x^m} \\[1ex]
		\vdots & \ddots & \vdots \\[0.5ex]
		\displaystyle\pdv{X^m}{\displaystyle x^1} & \cdots &
			\displaystyle\pdv{X^m}{\displaystyle x^m} ] (\V{x})
	=(\JacobiD{\V{x}})^{-1}(\V{X})
	=\mqty[
		\displaystyle\pdv{x^1}{\displaystyle X^1} & \cdots &
			\displaystyle\pdv{x^1}{\displaystyle X^m} \\[1ex]
		\vdots & \ddots & \vdots \\[0.5ex]
		\displaystyle\pdv{x^m}{\displaystyle X^1} & \cdots &
			\displaystyle\pdv{x^m}{\displaystyle X^m}
		]^{-1} (\V{X}) \fullstop
\end{equation}
用代数的方法总可以求出
\begin{equation}
	\pdv{X^s}{x^\alpha}(\V{x})=\varphi^s_\alpha(\V{X}) \comma
\end{equation}
其中的 $\varphi^s_\alpha$ 是通过矩阵求逆确定的函数。这样就有
\begin{equation}
	\pdv{f}{x^\alpha} (\V{x})
	=\sum_{s=1}^{m} \pdv{\tilde{f}}{\displaystyle X^s}
		\qty\big(\V{X}(\V{x})) \cdotp
		\varphi^s_\alpha (\V{x}) \fullstop
\end{equation}


%开集:任意一点吹出一个球,这个球上的每个点都落在区域内