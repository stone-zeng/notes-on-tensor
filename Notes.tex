%TODO 等的使用
%%TODO 内容问题
%%HACK 粗鄙技巧
%%CODE 代码改进
%%BUG  如题

%TODO——注意事项

\documentclass[oneside]{book}

%TODO——宏包
%% 方便定义命令(星号版本)
\usepackage{suffix}

%% 页面尺寸
\usepackage{geometry}
	\geometry{
		a4paper,
		left = 2.54 cm, right = 2.54 cm, top = 3.18 cm, bottom = 3.18 cm,
		headheight = 3 cm
	}

%% 设置标题
\usepackage{titlesec}

%% 常规字体选择
% 与 `graphicx' 冲突
\PassOptionsToPackage{draft}{graphicx}
\usepackage[no-math]{fontspec}
	\setmainfont{XITS}%[SlantedFont = XITS Italic]
	\setsansfont{TeX Gyre Heros}%[SlantedFont = TeX Gyre Heros Italic]
	\setmonofont{TeX Gyre Cursor}[Ligatures = NoCommon]

%% AMS 数学支持
\usepackage{amsmath}
	% 允许多行公式中间分页
	\allowdisplaybreaks
	% 调整多行公式间距
	\jot = 6pt
	% 调整公式前后间距
%	\AtBeginDocument{
%		\abovedisplayshortskip = \abovedisplayskip
%		\belowdisplayshortskip = \belowdisplayskip
%	}
\usepackage{amssymb}

%% 数学工具(Must before `unicode-math')
\usepackage{mathtools}

%% Pi 符号
\usepackage{pifont}
	% 正确 √
	\newcommand{\cmark}{\ding{51}}
	% 错误 ×
	\newcommand{\xmark}{\ding{55}}

%% 调用 Unicode OpenType 数学字体
\usepackage{unicode-math}
	\setmathfont[
		math-style = ISO,
		bold-style = ISO
	]{xits-math.otf}
%	]{texgyreschola-math.otf}
%	]{texgyredejavu-math.otf}
%	]{texgyrepagella-math.otf}
%	]{latinmodern-math.otf}
%	]{Asana-math.otf}
%	]{MinionMath-Regular.otf}
%	]{libertinusmath-regular.otf}
%	]{Cambria Math}
% 加粗使用 \symbf{}
% 直立希腊字母:\uppi 等

%% 中文文字处理
\usepackage[UTF8, heading = true]{ctex}
	\pagestyle{plain}
	\ctexset{
		section/format+ = {\normalfont\sffamily},
		subsection/format+ = {\normalfont\sffamily}
	}

%% XeTeX 下 CJK 文字处理
\usepackage{xeCJK}
	\setCJKmainfont[
		BoldFont = 方正黑体_GBK,
		ItalicFont = 方正楷体_GBK,
		Mapping = fullwidth-stop
	]{方正书宋_GBK}
	\setCJKsansfont[
		BoldFont = 方正黑体_GBK,
		ItalicFont = 方正黑体_GBK,
		Mapping = fullwidth-stop
	]{方正黑体_GBK}
	\setCJKmonofont[
		BoldFont = 方正仿宋_GBK,
		ItalicFont = 方正楷体_GBK,
		Mapping = fullwidth-stop
	]{方正仿宋_GBK}
	\setCJKfamilyfont{宋体}{方正书宋_GBK}
	\setCJKfamilyfont{楷体}{方正楷体_GBK}
	\setCJKfamilyfont{黑体}{方正黑体_GBK}
	\setCJKfamilyfont{仿宋}{方正仿宋_GBK}

%% LuaTeX 下 CJK 文字处理
%\usepackage{luatexja}
%\usepackage[no-math]{luatexja-fontspec}
%	\setmainjfont{FandolSong}
%	\setsansjfont{MS Gothic}
%	\setmainjfont{FZShuSong-Z01}
	%[
%		BoldFont = 方正黑体_GBK,
%		ItalicFont = 方正楷体_GBK,
%		Mapping = fullwidth-stop
%	]
	
%	\setsansjfont[
%		BoldFont = 方正黑体_GBK,
%		ItalicFont = 方正黑体_GBK,
%		Mapping = fullwidth-stop
%	]{方正黑体_GBK}
%%	\setCJKsansfont[
%%		BoldFont = 思源黑体 CN Regular,
%%		ItalicFont = 思源黑体 CN Regular,
%%		Mapping = fullwidth-stop
%%	]{思源黑体 CN Regular}
%	\setmonojfont[
%		BoldFont = 方正仿宋_GBK,
%		ItalicFont = 方正楷体_GBK,
%		Mapping = fullwidth-stop
%	]{方正仿宋_GBK}

%% 脚注增强版
\usepackage[stable, perpage, bottom]{footmisc}
	% 需要调用 pifont 宏包
	% 衬线加圈阳文数字:\ding{172}~\ding{181} (1~10)
	% 无衬线加圈阳文数字:\ding{192}~\ding{201} (1~10)
	\renewcommand{\thefootnote}{\ding{\numexpr191+\value{footnote} } }
	%TODO:20160705 脚注不用上标
	%HACK:20160709 见 http://tex.stackexchange.com/q/19844
	\makeatletter
	\newlength{\fnBreite}
	\renewcommand{\@makefntext}[1]{%
		\settowidth{\fnBreite}{\footnotesize\@thefnmark.i}%
		\protect\footnotesize\upshape%
		\setlength{\@tempdima}{\columnwidth}%
		\addtolength{\@tempdima}{-\fnBreite}%
		\makebox[\fnBreite][l]{\@thefnmark\phantom{  }}%
		\parbox[t]{\@tempdima}%
		{\everypar{\hspace*{1em}}\hspace*{-1em}\upshape#1}%
	}
	\makeatother

%% 颜色
\usepackage[svgnames]{xcolor}

%% 表格虚线
\usepackage{arydshln}

%% 图形
% 已被`fontspec' 调用
%\usepackage{graphicx}

%% 绘图
%\usepackage{tikz}
%	\usepgflibrary{arrows.meta}
%	\tikzset{>=Stealth}

%% 强调公式
\usepackage[ntheorem]{empheq}

%% 张量
\usepackage{tensor}

%% 物理、数学符号
\usepackage{physics}

%% 单位
\usepackage{siunitx}

%% 定制列表环境
\usepackage{enumitem}
	% 定义带缩进的左对齐格式
	% 前一个参数 0 em 确定标签的位置
	% 后一个参数 1.45 em 确定标签与文字的距离
	%HACK:20160924 与字体、字号、标签内容有关
	\SetLabelAlign{leftalignwithindent}%
		{\hspace{0 em} \makebox[1.45 em][l]{#1}}

%% 定理类环境
\usepackage[thmmarks, amsmath]{ntheorem}
	\theoremstyle{nonumberplain} %带编号
	\theoremheaderfont{\bfseries}
	\theorembodyfont{\normalfont}
	\theoremsymbol{\mbox{$\Box$}}
		\newtheorem{myProof}{证明:}

%% 索引
%\usepackage{imakeidx}
\usepackage{makeidx}
%	\makeindex%[options = {-s MyIndexStyle.mst}]
	\makeindex
%		name = Main,
%		title = {xxxxx Index},
%		intoc= true
%	]
	\newcommand{\idx}[1]{\index{#1}}
	%HACK:20170114 缺字,改用浪纹连接符代替
	\newcommand{\idxOmit}{\linkTilde}

%% 索引格式
\usepackage[
		hangindent = 6 em,
		subindent = 2 em,
		subsubindent = 4 em
	]{idxlayout}

%% 交叉引用,超链接等
\usepackage[hyperindex]{hyperref}
	\makeatletter
	\AtBeginDocument{
		\hypersetup{
			%% PDF 标题作者
			pdftitle = {\@title},
			pdfauthor = {\@author}
		}
	}
	\makeatother

	\hypersetup{
		%% PDF 书签
		bookmarksopen = true,
		bookmarksopenlevel = 1,
		bookmarksnumbered = true,
		%% 脚注
%			hyperfootnotes = false,
		%% 目录  只引用页码
		linktoc = page,
		%% 超链接边框
		pdfborder = 0 0 0,
		%% 超链接颜色
		colorlinks,
		linkcolor = {red!60!black},
		citecolor = {green!50!black},
		urlcolor = {blue!70!black}
	}
% 脚注引用
% 直接用 \ref
%\makeatletter
%\newcommand{\footnoteref}[1]{\protected@xdef%
%	\@thefnmark{\ref{#1}}\@footnotemark}
%\makeatother

%TODO——环境
%% 定制列表(编号)
\newenvironment{myEnum}
	{\enumerate[
		align = leftalignwithindent,
		% 条目第一段缩进
		itemindent = 2 em,
		% 条目第二段缩进
		listparindent = 2 em,
		% 文字左边距
		leftmargin = 0 pt,
		topsep = 0 pt,
		itemsep = 0 pt,
		parsep = 0 pt
	]}
	{\endenumerate}
%% 子公式
\newenvironment{mySubEq}
	{\subequations \renewcommand{\theequation}%
		{\theparentequation-\alph{equation}}}%
	{\endsubequations%
		\ignorespacesafterend}
%% 大括号公式
\newenvironment{braceEq}[1][align]
	{\mySubEq%
		\setkeys{EmphEqEnv}{#1}%
		\setkeys{EmphEqOpt}{left=\empheqlbrace}%
		\EmphEqMainEnv}%
	{\endEmphEqMainEnv \endmySubEq}
%% 大括号公式2(分情况讨论)
\newenvironment{braceEq*}[2][align]
	{\mySubEq%
		\setkeys{EmphEqEnv}{#1}%
		\setkeys{EmphEqOpt}{left={{\displaystyle #2}\empheqlbrace}}%
		\EmphEqMainEnv}%
	{\endEmphEqMainEnv \endmySubEq}

%TODO——命令
%% 浪纹连接号、代字符
% Unicode FF5E ~ FULLWIDTH TILDE
\newcommand{\linkTilde}{\symbol{65374}}
% Unicode 2053 ⁓ SWUNG DASH
% see \idxOmit
\newcommand{\swungDash}{\symbol{8275}}
%% 空行
%\newcommand{\blankline}{\mbox{}\par\mbox{}}
\newcommand{\blankline}{\mbox{}}
%% 强调
\newcommand{\emphA}[1]{{\bfseries #1}}
\newcommand{\emphB}[1]{{\itshape #1}}
%% 高亮(Highlight)
\newcommand{\hl}[2][yellow!60!white]{\mathchoice%
	{\colorbox{#1}{$\displaystyle #2$}}%
	{\colorbox{#1}{$\textstyle #2$}}%
	{\colorbox{#1}{$\scriptstyle #2$}}%
	{\colorbox{#1}{$\scriptscriptstyle #2$}}}
\newcommand{\hlw}[1]{\mathchoice%
	{\colorbox{white}{$\displaystyle #1$}}%
	{\colorbox{white}{$\textstyle #1$}}%
	{\colorbox{white}{$\scriptstyle #1$}}%
	{\colorbox{white}{$\scriptscriptstyle #1$}}}
\newcommand{\myPROBLEM}[2][\today]{\colorbox{red}{#1\quad#2}}

%TODO——数学命令
%--------数学字体--------%
%% 数集 黑板粗体
\let\SET=\symbb
%% 空间 花体
\let\SPACE=\symcal
%% 特殊空间 无衬线直立
\let\SPACEX=\symsfup
%% 域(领域,定义域) 哥特体
\let\DOMAIN=\symfrak
%% 符号 倾斜粗体
\let\SYMBOL=\symbf
%% 一般算子 无衬线直立
\let\OPERATOR=\symsfup
%% 特殊算子 无衬线直立粗体
\let\OPERATORX=\symbfsfup

%--------数集--------%
\newcommand{\realR}{\SET{R}}
\newcommand{\intN}{\SET{N}}
%% m 维实空间
\newcommand{\Rm}{\realR^m}
%% 自然数集(正整数集)
\newcommand{\natN}{\intN^{*}}

%--------空间--------%
%% 以 m 维实空间为底空间的(r)阶张量全体
\newcommand{\Tensors}[2][\Rm]{\SPACE{T}^{#2}\qty(#1)}
%% (r)阶置换全体
\newcommand{\Permutations}[1]{\SPACE{P}_{#1}}
%% 以 m 维实空间为底空间的(r)阶对称张量全体
\newcommand{\SymTensors}[2][\Rm]{\SPACE{S}^{#2}\qty(#1)}
%% 以 m 维实空间为底空间的(r)阶反对称张量全体
\newcommand{\SkwTensors}[2][\Rm]{\Lambda^{#2}\qty(#1)}
%% 线性变换全体
\newcommand{\LinearT}[2]{\SPACE{L}\qty(#1,\,#2)}

%--------特殊空间--------%
%% 正交矩阵全体(Orthogonal Matrices)
\newcommand{\Orth}{\SPACEX{Orth}}
%% 对称张量全体(Symmetric Tensors)
\newcommand{\Sym}{\SPACEX{Sym}}
%% 反对称张量全体(Anti-symmetric Tensors)
\newcommand{\Skw}{\SPACEX{Skw}}

%--------区域--------%
%% 邻域(Neighborhood)
\newcommand{\domB}[2]{\DOMAIN{B}_{#1}\qty(#2)}
%% 定义域(Domain)
\newcommand{\domD}[1]{\DOMAIN{D}_{#1}}

%--------量(Quantity)--------%
%% 向量(Vector)
\newcommand{\V}[1]{\SYMBOL{#1}}
%% 张量(Tensor)
\renewcommand{\T}[1]{\SYMBOL{#1}}
%% 张量分量(Tensor Components)
%\let\tc=\tensor
\newcommand{\tc}[2]{\tensor{#1}{#2}}
%% 矩阵(Matrix)
\newcommand{\Mat}[1]{\SYMBOL{#1}}
%% 置换(Permutation)
\newcommand{\Perm}[1]{\SYMBOL{#1}}

%--------特殊量--------%
%% Kronecker Delta
\newcommand{\KroneckerDelta}[2]{\delta^{#1}_{#2}}
%% Kronecker Delta (全下标形式)
\WithSuffix\newcommand\KroneckerDelta*[1]{\delta_{#1}}
%% Levi-Civita 记号
\newcommand{\LeviCivita}[1]{\tc{\epsilon}{#1}}
%% Eddington 张量
\newcommand{\EdTensor}{\T{\epsilon}}
%% 第一类 Christoffel 符号
\newcommand{\ChrA}[3]{\Gamma_{#1#2,\,#3}}
%% 第二类 Christoffel 符号
\newcommand{\ChrB}[3]{\tc{\Gamma}{^{#3}_{#1#2}}}
%% 单位正交基下的 Christoffel 符号
\newcommand{\ChrU}[3]{\Gamma\midscript{\orthIdx{#1#2#3}}}

%--------算子--------%
%% 置换算子(Permutation Operator)
\newcommand{\opPerm}[1][\Perm{\sigma}]{\OPERATOR{I}_{#1}}
%% 对称化算子(Symmetrizer)
\newcommand{\opSym}{\symcal{S}}
%% 反对称化算子(Antisymmetrizer)
\newcommand{\opSkw}{\symcal{A}}
%% Jacobi 矩阵(Differential Operator)
\DeclareMathOperator{\JacobiD}{\OPERATOR{D}\!}
%% 梯度算子(Gradient Operator)
\newcommand{\opGrad}{\T{\nabla}}
%% Laplace 算子(Laplace Operator)
\newcommand{\opLap}{\T{\nabla}^2}
%\newcommand{\opLap}{\symbfup{\Delta}}

%--------特殊算子--------%
%% 恒等映照(Identity)
\newcommand{\Id}{\OPERATORX{Id}}

%--------数学符号--------%
%% 映射(X: x → X)(Math Map)
\newcommand{\mmap}[3]{{#1}:\,{#2}\mapsto{#3}}
%% 带描述集合
\DeclarePairedDelimiterX\set[2]{\{}{\}}{#1 \;\delimsize\vert\; #2}
%% 中标
%% 见 http://tex.stackexchange.com/q/250961
\makeatletter
\newcommand{\mid@script}[2]{\vcenter{\hbox{$\m@th#1#2$}}}
\DeclareRobustCommand{\midscript}[1]{\mathchoice%
	{\mid@script\scriptstyle{#1}}%
	{\mid@script\scriptstyle{#1}}%
	{\mid@script\scriptscriptstyle{#1}}%
	{\mid@script\scriptscriptstyle{#1}}%
}
\makeatother
%% 单位正交基指标(Orthonormal Index)
\newcommand{\orthIdx}[1]{\langle{#1}\rangle}
%% 范数
\let\PHYSICSNORM=\norm
\renewcommand{\norm}[2][\Rm]{\abs{#2}_{#1}}
%% 线性分式
% 已有命令 \flatfrac
%\WithSuffix\newcommand\frac*[2]{\left.#1\middle/#2\right.}
%% 内积(括号形式)(Inner Product Braket)
\newcommand{\ipb}[3][\Rm]{\left\langle#2,\,#3\right\rangle_{#1}}
%% 张量积(Tensor Product)
\newcommand{\tp}{\otimes}
%% e-点积(e Dot Product)
\newcommand{\edp}[1][e]{\mathbin{\dbinom{#1}{\mathord{\cdot}}}}
%% 全点积(Full Dot Product)
\newcommand{\fdp}{\odot}
%% 点乘([Vector] Dot Product)
% 使用 \cdot 表示普通乘法,使用 \vdp 表示点乘
%% \dp 是 LaTeX 内部定义
\newcommand{\vdp}{\vysmblkcircle}
%% \cdot 作为 ordinary 符号
\newcommand{\cdotord}{\mathord{\cdot}}
%% 叉乘(Cross Product)(has been defined in `physics' package)
\renewcommand{\cp}{\vectimes}
%% 协变导数(Covariant Derivative)
\newcommand{\coD}[2]{\nabla_{\! #1}\,{#2}}
%% 逆变导数(Contravariant Derivative)
\newcommand{\ctrD}[2]{\nabla^{#1}\,{#2}}
%% C^p 微分同胚(Diffeomorphism)
\newcommand{\DiffMorp}[1][p]{\symcal{C}^{#1}}
%% 直至 p 阶偏导数连续的函数(Continuous Function)
\newcommand{\cf}[3][p]{\symcal{C}^{#1}\qty(#2;\,#3)}
%% 大 O 记号
\newcommand{\bO}[1]{\mathop{\mscrO\!}\qty(#1)}
%% 小 O 记号
\newcommand{\sO}[1]{\mathop{\mscro\!}\qty(#1)}
%% 小 O 记号(带上下标)
\WithSuffix\newcommand\sO*[2]{\mathop{
	\tc{\mscro}{#1} \!}\qty(#2)}
%% 向量形式小 O 记号
\newcommand{\sOv}[1]{\mathop{\mbfscro\!}\qty(#1)}
%% 向量形式小 O 记号(带上下标)
\WithSuffix\newcommand\sOv*[2]{\mathop{
	\tc{\mbfscro}{#1} \!}\qty(#2)}
%% 复合(Composite)
\newcommand{\comp}{\circ}
%% 定义等号(Definition Equal)
\newcommand{\defeq}{\triangleq}
%% 转置(Transpose)
\newcommand{\trans}{^{\mkern-1.5mu\mathsf{T}}}
%% 取逆并转置(Inverse and Transpose)
\newcommand{\invTrans}{^{-\mkern-1.5mu\mathsf{T}}}
%% 增量(Increment)
\newcommand{\incr}{\increment}

%--------函数名称--------%
%% 符号函数(Sign)
\DeclareMathOperator{\sgn}{sgn}

%--------公式杂项--------%
%% 带圈数字(Circle digit)
\newcommand{\circNum}[1]{%
	\ifnum #1=0%
		\mathord{\symbol{9450}}%
	\else%
		\mathord{\symbol{\numexpr9311+#1}}%
	\fi}
%% align 空格
\newcommand{\alspace}{\mathrel{\phantom{=}}}
%% 标点、文字
\newcommand{\comma}{\text{,}}
\newcommand{\fullstop}{\text{.}}
\newcommand{\semicolon}{\text{;}}
\newcommand{\const}{\text{const.}}
\newcommand{\res}{\text{res.}}
%\renewcommand{\intertext}[1]{\shortintertext{\textit{#1}}}

%TODO——标题页
\title{张量分析}
\author{复旦大学 谢锡麟} % 此处为全角空格
\date{\today}


\begin{document}
%	\frontmatter
%	\maketitle
%	
%	\tableofcontents
%	
%	\mainmatter
%	\chapter{张量的定义及表示}
%		\input{Chapters/Definition_Tensor}
%	
%	\chapter{张量的运算性质}
%		\input{Chapters/Algebric_Properties_of_Tensor}
%	
%	\chapter{微分同胚(曲线坐标系)}
%		\input{Chapters/Curvilinear_Coordinats}
%	
%	\chapter{张量场可微性}
%		\input{Chapters/Derivative_on_Tensor_Field}
%	
	\chapter{非完整基理论}
		\section{完整基与非完整基的概念}
在 \ref{sec:局部基}~节中,我们利用曲线坐标系 $\V{X}(\V{x})$
构造了 $\Rm$ 上的一组(局部协变)基
\begin{equation}
	\qty{\V{g}_i(\V{x})=\pdv{\V{X}}{x^i} (\V{x})}^m_{i=1}
	\subset\Rm \comma
\end{equation}
它们称为\emphA{完整基}。
与之对应,不是由曲线坐标系诱导的基,称为\emphA{非完整基}。

\begin{figure}[h]
	\centering
	\includegraphics{Images/Holonomic_Nonholonomic_Basis.PNG}
	\caption{完整基与非完整基}
	\label{fig:完整基与非完整基}
\end{figure}

如图~\ref{fig:完整基与非完整基},
$x^i$-线的\emphB{切向量}构成一组局部协变基
$\qty{\V{g}_i(\V{x})}^m_{i=1}$,
它和它的对偶 $\qty{\V{g}^i(\V{x})}^m_{i=1}$ 都是完整基。
除此以外,我们当然可以选取另外的基 $\qty{\V{g}_{(i)}(\V{x})}^m_{i=1}$
和 $\qty{\V{g}^{(i)}(\V{x})}^m_{i=1}$,它们不是由曲线坐标系诱导,
因而是非完整基。

\section{非完整基下的张量梯度} \label{sec:非完整基下的张量梯度}
下面我们来考察张量梯度在非完整基下的表达形式。
在 \ref{sec:张量场的梯度}~节中,我们已经推导出了张量场的(右)梯度:
\begin{equation}
	\qty\big(\T{\Phi}\tp\opGrad) (\V{x})
	\defeq \pdv{\T{\Phi}}{x^\mu} (\V{x})
		\tp \V{g}^\mu(\V{x})
	=\coD{\mu}{\tc{\Phi}{^i_j^k} (\V{x})} \,
		\V{g}_i (\V{x}) \tp \V{g}^j (\V{x})
		\tp \V{g}_k (\V{x}) \tp \V{g}^\mu (\V{x}) \fullstop
	\label{eq:张量梯度_非完整基}
\end{equation}
这是一个四阶张量,对应的张量分量可记作
\begin{equation}
	\tc{\qty\big(\T{\Phi}\tp\opGrad)}{^i_j^k_\mu} (\V{x})
	\coloneq \coD{\mu}{\tc{\Phi}{^i_j^k} (\V{x})} \fullstop
	\label{eq:张量梯度分量_非完整基}
\end{equation}
除此以外,其他的基当然也可以用来表示该张量,
比如前文提到过的 $\qty{\V{g}_{(i)}(\V{x})}^m_{i=1}$
和 $\qty{\V{g}^{(i)}(\V{x})}^m_{i=1}$,它们都是非完整基。

非完整基与完整基之间的关系,可以利用
\ref{subsec:相对不同基的张量分量之间的关系}~小节中引入的%
\emphB{坐标转换关系}来获得:
\begin{braceEq}
	\V{g}_{(i)} (\V{x})
		&= c^k_{(i)} (\V{x}) \, \V{g}_k (\V{x}) \comma \\
	\V{g}^{(i)} (\V{x})
		&= c_k^{(i)} (\V{x}) \, \V{g}^k (\V{x}) \semicolon \\
	\V{g}_i (\V{x})
		&= c^{(k)}_i (\V{x}) \, \V{g}_{(k)} (\V{x}) \comma \\
	\V{g}^i (\V{x})
		&= c_{(k)}^i (\V{x}) \, \V{g}^{(k)} (\V{x}) \fullstop
\end{braceEq}
\myPROBLEM[2017-01-20]{坐标转换关系}\\
其中的基转换系数都是已知量,它们的定义如下:\footnote{
	只有两个基转换系数的原因是内积具有交换律。}
\begin{braceEq}
	c^j_{(i)} (\V{x})
		&\coloneq \ipb{\V{g}_{(i)} (\V{x})}{\V{g}^j (\V{x})} \comma \\
	c_j^{(i)} (\V{x})
		&\coloneq \ipb{\V{g}^{(i)} (\V{x})}{\V{g}_j (\V{x})} \fullstop
\end{braceEq}
代入 \eqref{eq:张量梯度_非完整基}~式,可有\footnote{
	这里我们省略了“$(\V{x})$”。}
\begin{align}
	\T{\Phi}\tp\opGrad
	&=\coD{\mu}{\tc{\Phi}{^i_j^k}} \,
		\qty(\vphantom{\frac{0}{0}}
			\V{g}_i \tp \V{g}^j \tp \V{g}_k \tp \V{g}^\mu) \notag \\
	&=\coD{\mu}{\tc{\Phi}{^i_j^k} (\V{x})} \,
		\qty[\vphantom{\frac{0^0}{0^0}}
			\qty(c^{(p)}_i \, \V{g}_{(p)})
			\tp \qty(c_{(q)}^j \, \V{g}^{(q)})
			\tp \qty(c^{(r)}_k \, \V{g}_{(r)})
			\tp \qty(\vphantom{\frac{0}{0}}
				c_{(\alpha)}^\mu \, \V{g}^{(\alpha)})] \notag
	\intertext{根据线性性,提出系数:}
	&=\qty(c^{(p)}_i c_{(q)}^j c^{(r)}_k c_{(\alpha)}^\mu
			\coD{\mu}{\tc{\Phi}{^i_j^k}})
		\qty(\vphantom{\frac{0}{0}}
			\V{g}_{(p)} \tp \V{g}^{(q)} \tp \V{g}_{(r)}
			\tp \V{g}^{(\alpha)}) \notag
	\intertext{写成张量分量与简单张量“乘积”的形式,即为}
	&\eqcolon \tc{\qty\big(\T{\Phi}\tp\opGrad)}%
			{^{(p)}_{\!(q)}^{\!(r)}_{\!(\alpha)}} \,
		\qty(\vphantom{\frac{0}{0}}
			\V{g}_{(p)} \tp \V{g}^{(q)} \tp \V{g}_{(r)}
			\tp \V{g}^{(\alpha)}) \fullstop
\end{align}
这样,我们就获得了非完整基下张量梯度的表示。
再利用式~\eqref{eq:张量梯度分量_非完整基},可知
\begin{equation}
	\tc{\qty\big(\T{\Phi}\tp\opGrad)}%
		{^{(p)}_{\!(q)}^{\!(r)}_{\!(\alpha)}} \,
	=c^{(p)}_i c_{(q)}^j c^{(r)}_k c_{(\alpha)}^\mu \,
		\tc{\qty(\vphantom{0^0}
			\T{\Phi}\tp\opGrad)}{^i_j^k_\mu} \fullstop
	\label{eq:非完整基下的张量梯度分量}
\end{equation}
以上结果与 \ref{subsec:相对不同基的张量分量之间的关系}~小节
中的推导是完全一致的。

\section{非完整基的形式运算}
在 \ref{sec:非完整基下的张量梯度}~节中,
我们利用\emphB{坐标转换关系}获得了张量梯度在非完整基下的表示。
而在本节,我们将通过定义,建立所谓“形式理论”,
获得一套更统一、更连贯的表述。

\myPROBLEM[2017-01-31]{统一、连贯?}

首先需要给出一些定义。

\begin{myEnum}
\item \emphA{形式偏导数}:
\begin{equation}
	\pdv{x^{(\mu)}} \defeq c^m_{(\mu)} \pdv{x^m} \fullstop
	\label{eq:形式偏导数}
\end{equation}
注意 $\pdv*{x^{(\mu)}}$ 本身是不能用极限形式来定义的,
因为曲线坐标系中并不存在有 $x^{(\mu)}$ 坐标线。

\item \emphA{形式 Christoffel 符号}:
\begin{equation}
	\ChrB{(\alpha)}{(\beta)}{(\gamma)}
	\defeq c^i_{(\alpha)} c^j_{(\beta)} c^{(\gamma)}_k \ChrB{i}{j}{k}
		-c^i_{(\alpha)} c^j_{(\beta)} \pdv{c^{(\gamma)}_j}{x^i}
	=c^i_{(\alpha)} c^j_{(\beta)}
		\qty(c^{(\gamma)}_k \ChrB{i}{j}{k}
			-\pdv{c^{(\gamma)}_j}{x^i}) \fullstop
\end{equation}
\myPROBLEM[2017-01-31]{第一类形式 Christoffel 符号}

\item \emphA{形式协变导数}。我们以三阶张量 $\T{\Phi}$ 为例给出定义。
$\T{\Phi}$ 在非完整基下可以用混合分量表示如下:
\begin{equation}
	\tc{\Phi}{^{(\alpha)}_{\!(\beta)}^{\!(\gamma)}}
	\coloneq \T{\Phi} \qty(\V{g}^{(\alpha)},\,\V{g}_{(\beta)},\,
		\V{g}^{(\gamma)}) \fullstop
\end{equation}
它相对 $x^{(\mu)}$ 分量的形式协变导数为
\begin{equation}
	\coD{(\mu)}{\tc{\Phi}{^{(\alpha)}_{\!(\beta)}^{\!(\gamma)}}}
	\defeq \pdv{\tc{\Phi}{^{(\alpha)}_{\!(\beta)}^{\!(\gamma)}}}%
		{x^{(\mu)}}
	+\ChrB{(\mu)}{(\sigma)}{(\alpha)}
		\tc{\Phi}{^{(\sigma)}_{\!(\beta)}^{\!(\gamma)}}
	-\ChrB{(\mu)}{(\beta)}{(\sigma)}
		\tc{\Phi}{^{(\alpha)}_{\!(\sigma)}^{\!(\gamma)}}
	+\ChrB{(\mu)}{(\sigma)}{(\gamma)}
		\tc{\Phi}{^{(\alpha)}_{\!(\beta)}^{\!(\sigma)}} \fullstop
	\label{eq:形式协变导数}
\end{equation}
回顾 \ref{sec:张量场的偏导数_协变导数}~节,
\eqref{eq:协变导数定义}~式给出了完整基下协变导数的定义:
\begin{equation}
	\coD{\mu}{\tc{\Phi}{^i_j^k}} \defeq
	\pdv{\tc{\Phi}{^i_j^k}}{x^\mu}
	+\ChrB{\mu}{s}{i} \tc{\Phi}{^s_j^k}
	-\ChrB{\mu}{j}{s} \tc{\Phi}{^i_s^k}
	+\ChrB{\mu}{s}{k} \tc{\Phi}{^i_j^s} \fullstop
\end{equation}
可以看出\emphB{形式}协变导数的定义与它几乎一模一样。
\end{myEnum}

\blankline

接下来我们要证明
\begin{equation}
	\coD{(\mu)}{\tc{\Phi}{^{(\alpha)}_{\!(\beta)}^{\!(\gamma)}}}
	=c^m_{(\mu)} c^{(\alpha)}_i c^j_{(\beta)} c^{(\gamma)}_k \,
		\coD{m}{\tc{\Phi}{^i_j^k}} \fullstop
\end{equation}
代入式~\eqref{eq:张量梯度分量_非完整基} 和
\eqref{eq:非完整基下的张量梯度分量},即得
\begin{equation}
	\coD{(\mu)}{\tc{\Phi}{^{(\alpha)}_{\!(\beta)}^{\!(\gamma)}}}
	=\tc{\qty(\vphantom{0^0}
		\T{\Phi}\tp\opGrad)}{^{(p)}_{\!(q)}^{\!(r)}_{\!(\alpha)}}
	\fullstop
\end{equation}
换句话说,此处我们正是要验证这种“形式理论”与
\ref{sec:非完整基下的张量梯度}~节中坐标转换关系的一致性。

\begin{myProof}
左边按照 \eqref{eq:形式协变导数}~式展开,第一项为
\begin{align}
	\pdv{\tc{\Phi}{^{(\alpha)}_{\!(\beta)}^{\!(\gamma)}}}%
		{x^{(\mu)}}
	&=c^m_{(\mu)} \pdv{\tc{\Phi}{^{(\alpha)}_{\!(\beta)}%
			^{\!(\gamma)}}}{x^m} \notag
	\intertext{这里用到了形式偏导数的定义 \eqref{eq:形式偏导数}~式。
	然后利用坐标转换关系展开张量分量:}
	&=c^m_{(\mu)} \pdv{x^m}
		\qty(c^{(\alpha)}_i c^j_{(\beta)} c^{(\gamma)}_k
			\tc{\Phi}{^i_j^k}) \notag
	\intertext{按照通常的偏导数法则直接打开:}
	&=c^m_{(\mu)} c^j_{(\beta)} c^{(\gamma)}_k
			\pdv{c^{(\alpha)}_i}{x^m} \tc{\Phi}{^i_j^k}
		+c^m_{(\mu)} c^{(\alpha)}_i c^{(\gamma)}_k
			\pdv{c^j_{(\beta)}}{x^m} \tc{\Phi}{^i_j^k}
		+c^m_{(\mu)} c^{(\alpha)}_i c^j_{(\beta)}
			\pdv{c^{(\gamma)}_k}{x^m} \tc{\Phi}{^i_j^k} \notag \\*
	&\alspace{}
		+c^m_{(\mu)} c^{(\alpha)}_i c^j_{(\beta)} c^{(\gamma)}_k
			\pdv{\tc{\Phi}{^i_j^k}}{x^m} \notag \\
	&=c^m_{(\mu)} \tc{\Phi}{^i_j^k}
		\qty(c^j_{(\beta)} c^{(\gamma)}_k \pdv{c^{(\alpha)}_i}{x^m}
			+c^{(\alpha)}_i c^{(\gamma)}_k \pdv{c^j_{(\beta)}}{x^m}
			+c^{(\alpha)}_i c^j_{(\beta)} \pdv{c^{(\gamma)}_k}{x^m})
		\notag \\*
	&\alspace{}
		+c^m_{(\mu)} c^{(\alpha)}_i c^j_{(\beta)} c^{(\gamma)}_k
			\pdv{\tc{\Phi}{^i_j^k}}{x^m} \fullstop
	\label{eq:张量分量偏导数_非完整基形式运算}
\end{align}

接下来处理含有形式 Christoffel 符号的三项,分别是
\begin{mySubEq}
	\begin{align}
		\ChrB{(\mu)}{(\sigma)}{(\alpha)}
			\tc{\Phi}{^{(\sigma)}_{\!(\beta)}^{\!(\gamma)}}
		&=c^p_{(\mu)} c^q_{(\sigma)}
			\qty(c^{(\alpha)}_s \ChrB{p}{q}{s} - \pdv{c^{(\alpha)}_q}{x^p})
			\cdot \tc{\Phi}{^{(\sigma)}_{\!(\beta)}^{\!(\gamma)}} \notag \\
		&=c^p_{(\mu)} \hl{c^q_{(\sigma)}}
			\qty(c^{(\alpha)}_s \ChrB{p}{q}{s} - \pdv{c^{(\alpha)}_q}{x^p})
			\cdot \hl{c^{(\sigma)}_i} c^j_{(\beta)} c^{(\gamma)}_k
				\tc{\Phi}{^i_j^k} \notag
		\intertext{根据式~\eqref{eq:坐标转换系数的乘积},我们有
			$c^q_{(\sigma)} c^{(\sigma)}_i = \KroneckerDelta{q}{i}$,于是}
		&=c^p_{(\mu)} \tc{\Phi}{^i_j^k}
			\qty(c^{(\alpha)}_s c^j_{(\beta)} c^{(\gamma)}_k \ChrB{p}{i}{s}
				-c^j_{(\beta)} c^{(\gamma)}_k \pdv{c^{(\alpha)}_i}{x^p})
		\semicolon \\
		%
		-\ChrB{(\mu)}{(\beta)}{(\sigma)}
			\tc{\Phi}{^{(\alpha)}_{\!(\sigma)}^{\!(\gamma)}}
		&=-c^p_{(\mu)} c^q_{(\beta)}
			\qty(c^{(\sigma)}_s \ChrB{p}{q}{s} - \pdv{c^{(\sigma)}_q}{x^p})
			\cdot \tc{\Phi}{^{(\alpha)}_{\!(\sigma)}^{\!(\gamma)}} \notag \\
		&=-c^p_{(\mu)} c^q_{(\beta)}
			\qty(\hl{c^{(\sigma)}_s} \ChrB{p}{q}{s}
				-\pdv{c^{(\sigma)}_q}{x^p})
			\cdot c^{(\alpha)}_i \hl{c^j_{(\sigma)}} c^{(\gamma)}_k
				\tc{\Phi}{^i_j^k} \notag \\
		&=c^p_{(\mu)} \tc{\Phi}{^i_j^k}
			\qty(-c^{(\alpha)}_i c^q_{(\beta)} c^{(\gamma)}_k \ChrB{p}{q}{j}
				+c^{(\alpha)}_i c^q_{(\beta)} c^{(\gamma)}_k c^j_{(\sigma)}
					\pdv{c^{(\sigma)}_q}{x^p}) \semicolon \\
		%
		\ChrB{(\mu)}{(\sigma)}{(\gamma)}
			\tc{\Phi}{^{(\alpha)}_{\!(\beta)}^{\!(\sigma)}}
		&=c^p_{(\mu)} c^q_{(\sigma)}
			\qty(c^{(\gamma)}_s \ChrB{p}{q}{s} - \pdv{c^{(\gamma)}_q}{x^p})
			\cdot \tc{\Phi}{^{(\alpha)}_{\!(\beta)}^{\!(\sigma)}} \notag \\
		&=c^p_{(\mu)} \hl{c^q_{(\sigma)}}
			\qty(c^{(\gamma)}_s \ChrB{p}{q}{s} - \pdv{c^{(\gamma)}_q}{x^p})
			\cdot c^{(\alpha)}_i c^j_{(\beta)} \hl{c^{(\sigma)}_k}
				\tc{\Phi}{^i_j^k} \notag \\
		&=c^p_{(\mu)} \tc{\Phi}{^i_j^k}
			\qty(c^{(\alpha)}_i c^j_{(\beta)} c^{(\gamma)}_s \ChrB{p}{k}{s}
				-c^{(\alpha)}_i c^j_{(\beta)} \pdv{c^{(\gamma)}_k}{x^p})
		\fullstop
	\end{align}
\end{mySubEq}
以上三式都有公因子 $c^p_{(\mu)} \tc{\Phi}{^i_j^k}$。
为了进一步化简,不妨将哑标 $p$ 换为 $m$。这样可有
\begin{align}
	&\alspace \ChrB{(\mu)}{(\sigma)}{(\alpha)}
		\tc{\Phi}{^{(\sigma)}_{\!(\beta)}^{\!(\gamma)}}
	-\ChrB{(\mu)}{(\beta)}{(\sigma)}
		\tc{\Phi}{^{(\alpha)}_{\!(\sigma)}^{\!(\gamma)}}
	+\ChrB{(\mu)}{(\sigma)}{(\gamma)}
		\tc{\Phi}{^{(\alpha)}_{\!(\beta)}^{\!(\sigma)}} \notag \\
	&=c^m_{(\mu)} \tc{\Phi}{^i_j^k}
		\left[\vphantom{\pdv{c^{(\gamma)}_k}{x^m}} \qty(
			c^{(\alpha)}_s c^j_{(\beta)} c^{(\gamma)}_k \ChrB{m}{i}{s}
			-c^{(\alpha)}_i c^q_{(\beta)} c^{(\gamma)}_k \ChrB{m}{q}{j}
			+c^{(\alpha)}_i c^j_{(\beta)} c^{(\gamma)}_s \ChrB{m}{k}{s})
		\right. \notag \\
	&\alspace\phantom{c^m_{(\mu)} \tc{\Phi}{^i_j^k}\left[\right]}
		\left. {}-c^j_{(\beta)} c^{(\gamma)}_k \pdv{c^{(\alpha)}_i}{x^m}
			+c^{(\alpha)}_i c^q_{(\beta)} c^{(\gamma)}_k c^j_{(\sigma)}
				\pdv{c^{(\sigma)}_q}{x^m}
			-c^{(\alpha)}_i c^j_{(\beta)} \pdv{c^{(\gamma)}_k}{x^m}
		\right] \fullstop
\end{align}
该式与 \eqref{eq:张量分量偏导数_非完整基形式运算}~式相加,得
\begin{align}
	\coD{(\mu)}{\tc{\Phi}{^{(\alpha)}_{\!(\beta)}^{\!(\gamma)}}}
	&\defeq \pdv{\tc{\Phi}{^{(\alpha)}_{\!(\beta)}^{\!(\gamma)}}}%
		{x^{(\mu)}}
		+\ChrB{(\mu)}{(\sigma)}{(\alpha)}
			\tc{\Phi}{^{(\sigma)}_{\!(\beta)}^{\!(\gamma)}}
		-\ChrB{(\mu)}{(\beta)}{(\sigma)}
			\tc{\Phi}{^{(\alpha)}_{\!(\sigma)}^{\!(\gamma)}}
		+\ChrB{(\mu)}{(\sigma)}{(\gamma)}
			\tc{\Phi}{^{(\alpha)}_{\!(\beta)}^{\!(\sigma)}} \notag \\
	%
	&=c^m_{(\mu)} c^{(\alpha)}_i c^j_{(\beta)} c^{(\gamma)}_k
			\pdv{\tc{\Phi}{^i_j^k}}{x^m}
		+c^m_{(\mu)} \tc{\Phi}{^i_j^k} \left[
			\hl{c^j_{(\beta)} c^{(\gamma)}_k \pdv{c^{(\alpha)}_i}{x^m}}
			+c^{(\alpha)}_i c^{(\gamma)}_k \pdv{c^j_{(\beta)}}{x^m}
			+\hl[pink]{c^{(\alpha)}_i c^j_{(\beta)}
				\pdv{c^{(\gamma)}_k}{x^m}} \right. \notag \\*
	&\alspace
	\phantom{c^m_{(\mu)} c^{(\alpha)}_i c^j_{(\beta)} c^{(\gamma)}_k
			\pdv{\tc{\Phi}{^i_j^k}}{x^m}
		+c^m_{(\mu)} \tc{\Phi}{^i_j^k} \left[\right]}
		\left. {}
			+\qty(c^{(\alpha)}_s c^j_{(\beta)} c^{(\gamma)}_k \ChrB{m}{i}{s}
			-c^{(\alpha)}_i c^q_{(\beta)} c^{(\gamma)}_k \ChrB{m}{q}{j}
			+c^{(\alpha)}_i c^j_{(\beta)} c^{(\gamma)}_s \ChrB{m}{k}{s})
		\right. \notag \\*
	&\alspace
	\phantom{c^m_{(\mu)} c^{(\alpha)}_i c^j_{(\beta)} c^{(\gamma)}_k
			\pdv{\tc{\Phi}{^i_j^k}}{x^m}
		+c^m_{(\mu)} \tc{\Phi}{^i_j^k} \left[\right]}
		\left. {}
			-\hl{c^j_{(\beta)} c^{(\gamma)}_k \pdv{c^{(\alpha)}_i}{x^m}}
			+c^{(\alpha)}_i c^q_{(\beta)} c^{(\gamma)}_k c^j_{(\sigma)}
				\pdv{c^{(\sigma)}_q}{x^m}
			-\hl[pink]{c^{(\alpha)}_i c^j_{(\beta)}
				\pdv{c^{(\gamma)}_k}{x^m}} \right] \notag
	%
	\intertext{高亮部分相互抵消:}
	&=c^m_{(\mu)} c^{(\alpha)}_i c^j_{(\beta)} c^{(\gamma)}_k
			\pdv{\tc{\Phi}{^i_j^k}}{x^m}
		+c^m_{(\mu)} \tc{\Phi}{^i_j^k}
		\left[\vphantom{\pdv{c^{(\gamma)}_k}{x^m}} \qty(
			c^{(\alpha)}_s c^j_{(\beta)} c^{(\gamma)}_k \ChrB{m}{i}{s}
			-c^{(\alpha)}_i c^q_{(\beta)} c^{(\gamma)}_k \ChrB{m}{q}{j}
			+c^{(\alpha)}_i c^j_{(\beta)} c^{(\gamma)}_s \ChrB{m}{k}{s})
		\right. \notag \\
	&\alspace
	\phantom{c^m_{(\mu)} c^{(\alpha)}_i c^j_{(\beta)} c^{(\gamma)}_k
			\pdv{\tc{\Phi}{^i_j^k}}{x^m}
		+c^m_{(\mu)} \tc{\Phi}{^i_j^k} \left[\right]}
		\left. {}
			+c^{(\alpha)}_i c^{(\gamma)}_k \pdv{c^j_{(\beta)}}{x^m}
			+c^{(\alpha)}_i c^q_{(\beta)} c^{(\gamma)}_k c^j_{(\sigma)}
				\pdv{c^{(\sigma)}_q}{x^m} \right]
	\label{eq:非完整基形式理论推导}
\end{align}
注意到 $c^j_{(\beta)} = c^q_{(\beta)} \KroneckerDelta{j}{q}
	=c^q_{(\beta)} c^j_{(\sigma)} c^{(\sigma)}_q$,因此
\begin{equation}
	\pdv{c^j_{(\beta)}}{x^m}
	=\pdv{x^m} \qty(c^q_{(\beta)} c^j_{(\sigma)} c^{(\sigma)}_q)
	=c^j_{(\sigma)} c^{(\sigma)}_q \pdv{c^q_{(\beta)}}{x^m}
		+c^q_{(\beta)} c^{(\sigma)}_q \pdv{c^j_{(\sigma)}}{x^m}
		+c^q_{(\beta)} c^j_{(\sigma)} \pdv{c^{(\sigma)}_q}{x^m} \fullstop
\end{equation}
所以 \eqref{eq:非完整基形式理论推导}~式中最后一步的第二行就能够写成
\begin{align}
	&\alspace c^{(\alpha)}_i c^{(\gamma)}_k \pdv{c^j_{(\beta)}}{x^m}
		+c^{(\alpha)}_i c^q_{(\beta)} c^{(\gamma)}_k c^j_{(\sigma)}
			\pdv{c^{(\sigma)}_q}{x^m} \notag \\
	&=c^{(\alpha)}_i c^{(\gamma)}_k
		\qty(\pdv{c^j_{(\beta)}}{x^m}
			+c^q_{(\beta)} c^j_{(\sigma)} \pdv{c^{(\sigma)}_q}{x^m})
		\notag \\
	&=c^{(\alpha)}_i c^{(\gamma)}_k \qty(
			\hl{c^j_{(\sigma)}} c^{(\sigma)}_q \pdv{c^q_{(\beta)}}{x^m}
			+\hl[pink]{c^q_{(\beta)}} c^{(\sigma)}_q
				\pdv{c^j_{(\sigma)}}{x^m}
			+c^q_{(\beta)} \hl{c^j_{(\sigma)}} \pdv{c^{(\sigma)}_q}{x^m}
			+\hl[pink]{c^q_{(\beta)}} c^j_{(\sigma)}
				\pdv{c^{(\sigma)}_q}{x^m}) \notag
	\intertext{合并同类项:}
	&=c^{(\alpha)}_i c^{(\gamma)}_k \qty[
			c^j_{(\sigma)}
			\qty(c^{(\sigma)}_q \pdv{c^q_{(\beta)}}{x^m}
				+c^q_{(\beta)} \pdv{c^{(\sigma)}_q}{x^m})
		+c^q_{(\beta)}
			\qty(c^{(\sigma)}_q \pdv{c^j_{(\sigma)}}{x^m}
				+c^j_{(\sigma)} \pdv{c^{(\sigma)}_q}{x^m}) ] \notag \\
	&=c^{(\alpha)}_i c^{(\gamma)}_k \qty[
			\vphantom{\pdv{c^q_{(\beta)}}{x^m}}
			c^j_{(\sigma)} \pdv{x^m} \qty(c^{(\sigma)}_q c^q_{(\beta)})
			+c^q_{(\beta)} \pdv{x^m} \qty(c^{(\sigma)}_q c^j_{(\sigma)}) ]
		\notag
	\intertext{再次利用式~\eqref{eq:坐标转换系数的乘积},可得}
	&=c^{(\alpha)}_i c^{(\gamma)}_k \qty(
			c^j_{(\sigma)} \pdv{\KroneckerDelta{\sigma}{\beta}}{x^m}
			+c^q_{(\beta)} \pdv{\KroneckerDelta{j}{q}}{x^m} )
	=0 \fullstop
\end{align}
代回式~\eqref{eq:非完整基形式理论推导},有
\begin{align}
	\coD{(\mu)}{\tc{\Phi}{^{(\alpha)}_{\!(\beta)}^{\!(\gamma)}}}
	&=c^m_{(\mu)} c^{(\alpha)}_i c^j_{(\beta)} c^{(\gamma)}_k
			\pdv{\tc{\Phi}{^i_j^k}}{x^m}
		+c^m_{(\mu)} \tc{\Phi}{^i_j^k}
		\qty(\vphantom{\pdv{c^{(\gamma)}_k}{x^m}}
			c^{(\alpha)}_s c^j_{(\beta)} c^{(\gamma)}_k \ChrB{m}{i}{s}
			-c^{(\alpha)}_i c^q_{(\beta)} c^{(\gamma)}_k \ChrB{m}{q}{j}
			+c^{(\alpha)}_i c^j_{(\beta)} c^{(\gamma)}_s \ChrB{m}{k}{s})
		\notag \\
	&=c^m_{(\mu)} c^{(\alpha)}_i c^j_{(\beta)} c^{(\gamma)}_k
			\pdv{\tc{\Phi}{^i_j^k}}{x^m}
		+c^m_{(\mu)}
		\qty(\vphantom{\pdv{c^{(\gamma)}_k}{x^m}}
			c^{(\alpha)}_s c^j_{(\beta)} c^{(\gamma)}_k
				\ChrB{m}{i}{s} \tc{\Phi}{^i_j^k}
			-c^{(\alpha)}_i c^q_{(\beta)} c^{(\gamma)}_k
				\ChrB{m}{q}{j} \tc{\Phi}{^i_j^k}
			+c^{(\alpha)}_i c^j_{(\beta)} c^{(\gamma)}_s
				\ChrB{m}{k}{s} \tc{\Phi}{^i_j^k}) \notag
	\intertext{下面要对哑标进行重排。
		括号里的第一项:$s \leftrightarrow i$;
		第二项:$j \rightarrow s, q \rightarrow j$;
		第三项:$s \leftrightarrow k$。于是}
	&=c^m_{(\mu)} c^{(\alpha)}_i c^j_{(\beta)} c^{(\gamma)}_k
			\pdv{\tc{\Phi}{^i_j^k}}{x^m}
		+c^m_{(\mu)}
		\qty(\vphantom{\pdv{c^{(\gamma)}_k}{x^m}}
			c^{(\alpha)}_i c^j_{(\beta)} c^{(\gamma)}_k
				\ChrB{m}{s}{i} \tc{\Phi}{^s_j^k}
			-c^{(\alpha)}_i c^j_{(\beta)} c^{(\gamma)}_k
				\ChrB{m}{j}{s} \tc{\Phi}{^i_s^k}
			+c^{(\alpha)}_i c^j_{(\beta)} c^{(\gamma)}_k
				\ChrB{m}{s}{k} \tc{\Phi}{^i_j^s}) \notag \\
	&=c^m_{(\mu)} c^{(\alpha)}_i c^j_{(\beta)} c^{(\gamma)}_k
		\qty(\pdv{\tc{\Phi}{^i_j^k}}{x^m}
			+\ChrB{m}{s}{i} \tc{\Phi}{^s_j^k}
			-\ChrB{m}{j}{s} \tc{\Phi}{^i_s^k}
			+\ChrB{m}{s}{k} \tc{\Phi}{^i_j^s}) \notag \\
	&=c^m_{(\mu)} c^{(\alpha)}_i c^j_{(\beta)} c^{(\gamma)}_k \,
		\coD{m}{\tc{\Phi}{^i_j^k}} \fullstop
\end{align}
这就完成了证明。
\end{myProof}
%	
%	\chapter{曲线上的标架及其运动方程}
%		\input{Chapters/Curve_Frame}
%	
%	\backmatter
%	{
%%		\small
%%		\bibliography{Reference}
%		\printindex
%	}
\end{document}